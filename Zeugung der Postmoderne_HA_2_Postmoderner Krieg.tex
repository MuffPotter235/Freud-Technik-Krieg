\documentclass[11pt,a4paper,oneside,numbers=noenddot,bibliography=totocnumbered,DIV=13]{scrartcl}
%\usepackage[lmargin=2cm,rmargin=4cm,tmargin=3cm,bmargin=3cm]{geometry}
%Sprache
\usepackage[utf8]{inputenc}
\usepackage[T1]{fontenc}
\usepackage[ngerman]{babel}
%Typografie
\usepackage[bitstream-charter]{mathdesign}
%\usepackage{txfonts}
\usepackage[activate={true,nocompatibility},final,tracking=true,kerning=true,spacing=true,factor=1100,stretch=10,shrink=10]{microtype}
\usepackage{url}
%überschriften in Helvetica
\usepackage{sectsty}
\allsectionsfont{\fontfamily{phv}\selectfont\normalsize} 
%Schusterjungen
\clubpenalty = 10000 
\widowpenalty = 10000 
\displaywidowpenalty = 10000
% Zeilenabstand
\usepackage{setspace}
\onehalfspacing
%Zitate
\newcommand{\block}[2]{{\footnotesize\singlespacing\blockquote{\enquote{#1}{#2}}}\vspace{+2mm}}% <-- Hier kannst du den Abstand veraendern
\usepackage{tocstyle} 
%\usetocstyle{KOMAlike}%Find ich eigentlich gut 
\addtocontents{toc}{\protect\fontfamily{phv}} 


\setkomafont{title}{\normalfont\bfseries}


\begin{document}
\section{Post-Moderner Krieg}
In diesem Abschnitt soll näher auf die Entwicklung vom modernen zum post-modernen Krieg sowie auf die Charakteristika des post-modernen Krieges   eingegangen werden. Einen wichtigen Ausgangspunkt zur Beschreibung dessen was in diesem Text als post-modernen Krieg verstanden wird, bietet James der Derians Konzept des \textit{Virtous War}. Der Derian argumentiert das Kriege industrieller Staaten in der Post-Vietnam, Post-Kalter Krieg Ära zunehmend von zwei Kennzeichen geprägt sind: Zum einen werden post-moderne Kriege als \textit{jus bellum} dargestellt - als gerechter Krieg und notwendigen Einsatz militärischer Gewalt zum Schutz von zivilen Opfern. Zum anderen wird zum erreichen der militärischen Ziele ein \textit{technological fix} angewendet. In post-modernen Kriegen findet sich der Einsatz neuer Technologie um einen vermeintlich sauberen Krieg zu ermöglichen, in dem kaum eigne Verluste sowie zivile Tote zu beklagen sind. Beide Charakteristika haben die Konzept der (post-)modernen Kriegsführung nachhaltig Verändert.(\cite{DerDerian2001}) \\
Dabei gilt es zu beachten, das die post-moderne Kriege selten in ihrer Reinform auftauchen, ein Prototyp findet sich möglicherweise in der NATO Intervention gegen Serbien von März bis Juni 1999. Allerdings finden sich in den meisten Fällen Mischformen, in denen die Charakteristika die hier dem post-modernen Krieg zugeschrieben werden, nur sehr begrenzt zum tragen kommen. Der post-moderne Krieg muss daher eher als Entwicklung hin zu etwas, als Tendenz denn als alleinige Realität der Kriegsführung verstanden werden. Des weiteren ist der hier beschriebene \textit{Virtous War} eher in der Kriegsführung industrieller bzw. post-industrieller Staaten zu finden, da die Möglichkeit einen \textit{technological fix} einzuführen, (noch) stark von den ökonomischen Möglichkeiten eines Staates abhängig ist. Zwar finden sich Stimmen, die argumentieren, das auch Kriege bspw. zwischen bewaffnetet Gruppen zunehmend die Form eines \textit{Virtuose War} annehmen, allerdings finden sich doch noch signifikante unterscheide zwischen den technischen Möglichkeiten westlicher Industriestaaten und nicht-staatlichen Akteuren in der Kriegsführung.\\ 
\subsection{Vom modernen zum post-modernen Krieg}
%Post-Moderne als Intellektuelle Strömung
Christopher Coker sieht in der um die 1910/1920 entstehenden Anfängen der post-modernen Strömung, eine intellektuelle Unzufriedenheit mit der von der Moderne vertretenen Vorstellung einer klar geordneten, wissenschaftlich zugänglichen und widerspruchsfreien Welt. Die Post-Moderne als intellektuelle Strömung, begreift die Welt als fragmentiert, uneindeutig und widersprüchlich. Gleichzeitig wird in der Post-Moderne den Subjekten dieser Welt auch eine größere \textit{agency} zugeschrieben als dies in der Moderne der Fall wahr.(\cite{coker_collision_2012} S.4)\\
Bezogen auf den post-modernen Krieg argumentiert Coker, dass dieser sich sich ebenfalls durch eine Auflösung eines großen Meta-Narratives auszeichnet, das während moderner Kriege dazu benutzt werden konnte die Bevölkerung zu mobilisieren.\footnote{ebd.}\\
Edward Luttenwak argumentiert in seinem viel beachteten Artikel \textit{A post-Heroic Military Policy}, dass vor dem kalten Krieg, Kriegseintritte zumeist mit einer schnellen Mobilmachung der Bevölkerung einhergingen, während nach Beendigung der Feinseligkeiten eine ebenso schnelle De-Mobilisierung erfolgte, die das Militär auf den prä-Kriegs Zustand zurück brachte. Luttenwak argumentiert weiter, dass sich diese im Kalten Krieg geändert hat, da hier eine ständige Bereitschaft an starken militärischen Kräften erforderlich war, um eine wirksame Abschreckung zu gewährleisten. Allerdings sind insbesondere nach Ende des Vietnamkrieges, aufgrund von politischen Beschränkungen, nur wenige dieser Kräfte für tatsächliche Kampfhandlungen, jenseits der direkten Landesverteidigung einsetzbar.(\cite{Luttwak1996} S. 34ff.)\\
Daher stellt der Vietnamkrieg eine Zäsur in der Entwicklung von der modernen zur post-modernen Kriegsführung dar. Während die Streitkräfte der USA in der eignen Bevölkerung nach dem Sieg über die Achsenmächte und dem Ende des zweiten Weltkrieges enorme Popularität und Unterstützung genossen, wandelte sich dieses Bild während des Vietnamkrieges. Die hohen Verluste sowie der Rückzug der US-Streitkräfte aus Vietnam wurde in der Bevölkerung und auch im Militär selber als Niederlage wahrgenommen. Dies führte zu einer Phase der Reflexion und des Umdenkens sowohl im Militär als auch in der US-Amerikanischen Bevölkerung.[\cite{olson_war_2007} S.563 f.]\\
integraler Bestandteil der \textit{Containment} Strategie, als Prinzip der  amerikanischer Außenpolitik, war es die Ausbreitung des Einflussbereichs "fremder Mächte", speziell der Sowjetunion und anderer "sozialistischer"\footnote{"sozialistischer" meint hier in Konkurrenz zu der im Westen dominanten Kapitalismusform stehend; Ziel ist nicht eine Debatte darüber anzuregen ob und in wie weit das polit--ökonomische System der Sowjetunion, Chinas und anderer tatsächlich einem sozialistischen Gedanken entspricht} Staaten einzudämmen ohne dabei tatsächlich eine militärische Auseinandersetzung größeren Ausmaßes mit den betreffenden Staaten führen zu müssen.(\cite{olson_war_2007}S.566 ff.)\\         
Mit dem Scheitern der Strategie, bewaffnete Konflikte zu vermeiden und unter dem Eindruck von massiven eignen Verlusten, auf "Nebenkriegsschauplätzen", jenseits einer eigentlichen Konfrontation mit der Sowjetunion, geriet auch die \textit{Containment} Doktrin zunehmend unter Druck. Es kam erstmals seit Beginn des Kalten Krieges zu einer breiten Debatte um Außenpolitik.\footnote{ebd.}\\
Mit dem Ende des Kalten Krieges und der Zunahme von sogenannten \textit{Operations other than War}, wie Humanitären Interventioen oder Peacekeeping Einsätzen schon vor Ende des Kalten Krieges befand sich der Konsens bezüglich der \textit{Containment} Strategie der USA endgültig in Auflösung.(\cite{coker_postmodern_2008}S.7 f.)\\
Luttwaks Argument setzt genau an dieser Stelle an. Während, wie bereits erwähnt, im Kalten Krieg massive Anstrengungen unternommen wurden um ein quantitativ starkes Militär als wirksame Abschreckung gegen bewaffnete Konflikte unterhalb der Schwelle eines Nuklearkriegs zu haben, sind eben jene personenstarken Teilbereiche der Streitkräfte besonderen ineffektiv für den tatsächlichen Einsatz, da sie in Kampfhandlungen häufig auf kurze Distanz eingesetzt werden müssen und somit Gefahr laufen hohe eigne Verluste zu erleiden. Luttwak weist darauf hin das, insbesondere in Demokratien, hohe eigne Verluste meist zu einer errosionsrtigen Abnahme der öffentlichen Unterstützung für militärische Einsätze nach sich ziehen. (\cite{Luttwak1996}S.35 f.)\\
Ausdruck eben jener Furcht vor dem Verlust der öffentlichen Unterstützung bei zu hohen eigenen Verlusten stellt die sogenannte \textit{Weinberger Doctrine} (auch \textit{Weinberger-Powell} dar. Diese 1984 Doktrin zum Einsatz von militärischen Kräften durch die USA, reflektiert dabei die Erfahrungen aus Vietnam, aber auch die hohen Verluste der USA während des Peacekeeping Einsatzes der \textit{Multinational Force in Lebanon} 1982-1984 in Beirut. Weinberger fordert den Einsatz von Streitkräften auf ein minumum zu begrenzen und diese nur einzusetzen wenn sich die Regierung der öffentliche Unterstützung sicher sein könne.(\cite{Weinberger28.11.1984})  Die Weinberger Doktrin nimmt auch eine Beobachtung vorweg, die sich später sowohl in Cokers Charakterisierung des post-modernen Krieges als auch in Der Derians Beschreibung des \textit{Virtuous War} finden lässt - die zunehmende Durchmischung von Frieden und Krieg. Coker nennt es Ironie, dass während des Peacekeeping Einsatzes im Libanon mehr US Soldaten starben als während mancher Kampfhandlung (\cite{coker_postmodern_2008}S.7 ff.), während Der Derian eher auf den Aspekt der mit militärischen Drohungen garnierten Diplomatie hinweist.(\cite{DerDerian2001}S.198)\\
% Übergang verbessern
Das Ende des Kalten Krieges stellte dabei nicht nur den Beginn einer neuen militärisch/aussenpolitischen Ära da, sondern wurde in den USA auch als ein Übergang von einer Bi- zu einer Multipolaren Weltordnung bewertet, in der die bisherigen Strategien zur Abschreckung eines potentiellen Gegners, nur begrenzt geigten sind. Gleichzeitig  kam es unter der Clinton-Administration zu einer endgültigen Abkehr von der, inzwischen Überholten, \textit{Containment} Strategie und der Einführung einer idealistischeren Außenpolitik. Das Konzept des \textit{Enlargement}, das auf eine Ausbreitung von liberaler Marktwirtschaft und Demokratie setzt und dabei auch eher geneigt ist militärische Interventionen nach dem Prinzip des \textit{jus bellum} im Sinne einer Humanitären Intervention einzusetzen. (\cite{DerDerian2001}S.17 ff.)\\ Gerade das Prinzip des \textit{jus bellum} spielt hier eine große Rolle da es den Einsatz militärischer Gewalt legitimieren kann, indem es spezifische Bedingungen definiert, unter denen der Einsatz von Gewalt gerechtfertigt ist (vgl. beispielsweise:\cite{rengger_just_2002}) So bemerkt beispielsweise Banta in seinem Artikel, dass seit dem Ende des Kalten Krieges keine liberale Demokratie militärische Gewalt eingesetzt hat, ohne sich zumindest teilweise auf eine humanitäre Begründung zu stützen (\cite{banta_virtuous_2011}S.280 ff.).    
\subsection{Charakteristika des Post-Modernen Kriegs}
Ausgangspunkt der Betrachtung der Charakteristika des post-modernen Krieges ist die im vorherigen Abschnitt dargelegte veränderte Situation nach dem Ende des Kalten Krieges. Insbesondere die Einführung des \textit{Enlargements} Konzepts führte dabei zu einem stärkeren Einsatz militärischer Gewalt, zumeist in Verbindung mit sogenannten \textit{Operations other than War} wie "humanitären Intervention" oder Peackeeping-Einsätzen. Im Konzept des \textit{Enlargments} angelegt ist auch eine Legitimation dieser Gewaltanwendung zum Zweck der Ausbreitung von Konzepten wie Demokratie oder dem Schutz von Menschenrechten, innerhalb moderner \textit{jus bellum} Konzepte wie der \textit{Responsibility to Protect}.\\
Das \textit{jus bellum} Konzept ist dabei ein zentraler Punkt für den post-modernen Krieg westlicher Industrienationen. Nach der Auflösung der \textit{Containment} Strategie nah dem Wegfall der Sowjetunion als primäre Bedrohung wurde auch eine auf starken konventionellen und nuklear-Streitkräften beruhende Abschreckungsdoktrin mit der Vorbereitung auf größere Konflikte mit dem Gegner, beispielsweise die Planungen für eine Militärische Auseinandersetzung am sogenannten "Fulda Gap"(\cite{olson_war_2007}S. 566) zur Eindämmung des gegnerischen Machtbereichs,zunehmend nutzlos. Dagegen findet sich im \textit{Enlargement} die tatsächliche Anwendung militärischer Gewalt gegen verschiedene potentielle \textit{states of concern} legitimiert durch das \textit{jus bellum} Konzept (\cite{DerDerian2001}S. I ff.). Hier findet sich auch eine klare Abkehr von der in \textit{Weinberger Doktrin} vertretenen Position das der Einsatz von US Truppen auf ein Minimum beschränkt werden soll, stattessen wird zunehmend das Konzept eine \textit{liberalen Interventionismus} betrieben (\cite{lafeber_rise_2009}).  
%jus Bellum
Das \textit{jus bellum} Konzept stellt eine Denk-Tradition dar, die ihren beginn im Aufkommen des Christentums und des damit Verbundenen früh-christlichen Radikalpazifismus hat. In dieser Zeit, stellte \textit{jus bellum} eine Möglichkeit dar, dieses Gewaltverbot zu umgehen indem es die Anwendung militärischer Gewalt unter bestimmten Bedingungen, als Vereinbar mit dem christlichen Glauben ansah. \textit{Jus bellum} war zwar schon in seinen Anfängen ein säkulares Konzept, allerdings hat sich seit dem 17 Jhrd. die Debatte weiter ausdifferenziert und liegt und sich von seiner Gestalt als ethisch-Moralische Handlungsmaxime gelöst, so das es heute vor allem in seiner verrechtlichen Form, beispielsweise in Artikel 7 der UN Charter oder im internationalem Kriegsvölkerrecht vorliegt. Aktuell findet sich das \textit{jus bellum} Konzept auch in der vor einigen Jahren diskutierten \textit{Responsibility to Protect} (für \textit{jus bellum} vgl.: \cite{2002} S. 926 ff.; siehe auch: \cite{bellamy_responsibilities_2008} für \textit{Responsibility to Protect} siehe: \cite{international_commission_on_intervention_and_state_sovereignty_responsibility_2001}).\\
Eine neuere Debatte um das \textit{jus bellum} entstand im Verlaufe der 60er, 70er und 80er vor dem Hintergrund des Vietnam Krieges. Johnson argumentiert das während dieser Debatte primär ein Kriterienkatalog herangezogen wurde um zu prüfen, ob der Einsatz militärischer Gewalt im spezifischen Fall legitim ist bzw. war (\cite{johnson_contemporary_2013} S. 31 ff)).
Dazu werden zumeist zwei Bereiche des \textit{jus bellum} Konzept betrachtet: Einerseits das \textit{jus ad bellum} und andererseits das \textit{jus in bello}. 
\begin{quote}
“[…] beginning with the classic criteria of sovereign authority, just cause, and right intention, including the end of peace; then adding the prudential criteria widely applied today—reasonable hope of success, proportionality of ends, and lack of reasonable alternatives (last resort); and finally defining right conduct in war in terms of discrimination and proportionality of means.” (\cite{johnson_contemporary_2013}S. 32) \\
\end{quote}
\textit{Jus ad bellum} meint daher die Kriterien die vor dem Einsatz militärischer Gewalt gegeben sein müssen um diese als legitim zu sehen. Hierbei handelt es sich vor allem um den \textit{just cause} welcher vor allem den Schutz von "Unschuldigen" vor "Bösem", aus altruistischen Gründen, mit der Gewaltanwendung als \textit{last resort} und dem Ziel einen Frieden nach dem Ende der Gewaltanwendung wieder herzustellen.
 \textit{Jus in bello} bewertet das Verhalten der intervenierenden Seite im Konflikt selber und spiegelt sich auch beispielsweise im Kriegsvölkerrecht wieder. Dies heist vor allem der Schutz von Zivilisten und die Verhältnismäßigkeit bei Einsatz von Gewalt (\cite{rengger_just_2002}S.358 ff.). Aktuell ist vor allem eine Debatte, ob das Konzept um einen weiteren Teilbereich ergänzt werden muss, der das Verhalten nach Ende der Gewaltanwendung regelt. Die Notwendigkeit eines \textit{jus post bellum} ist allerdings umstritten (\cite{bellamy_responsibilities_2008}; siehe auch: \cite{banta_virtuous_2011}).\\    
%Sichtbarkeit des Schlachtfeldes
Hinzu kommt, dass so Christopher Coker,das neue Medien, allen voran das Fernsehen und das Internet auch für eine neue Sichtbarkeit des Krieges stoßen. Insbesondere zivile Opfer werden durch neue Medien besser sichtbar und Zwischenfälle können immer wieder Debatten über die Legitimität der militärischen Gewaltanwendung auslösen. Während sich dieser Trend schon zur Zeit des Vietnamkrieges feststellen lässt, hat er mit der breiten Einführung des Internets nochmal zugenommen. Eigene und zivile Verluste werden besser sichtbar und kommen, zumindest theoretisch, ohne die \textit{Gatekeeper-Funktion} etablierter Medien aus.\\
Diesbezüglich geht Cocker auch von einem einer Auflösung eine in sich selbst begründeten Narratives des Krieges aus und setzt an seine stelle viele fraktionaliserte Sichtweisen (\cite{coker_collision_2012} S.4 f.):
\begin{quote}
The image of civilians recovering from a mistimed airstrike; the image of children in the
rubble of a village—all diminish reputations by exposing the illegitimacy of power, or its use. War may still be
necessary but it is now no longer redeeming. (\cite{coker_collision_2012}S.5) 
\end{quote}  
Während Coker daher eine bessere Sichtbarkeit des Schlachtfeldes durch neue Medien feststellt, stellt er gleichzeitig fest das es seit dem Vietnam Krieg auch eine Gegenbewegung, innerhalb aber auch außerhalb des Militärs gibt, um die zerstörerische Natur des Krieges zu verbergen. Dies geschieht, so Coker, vor allem durch die Verwendung von neuen Sprachregelungen wie beispielsweise \textit{Colleteral Damage} (\cite{coker_postmodern_2008}S. 8 f.)\\  
%Der Derians Virtouse War
Vor dem Hintergrund der oben beschriebenen Veränderungen entwickelt auch James der Derian eine Hälfte seines Argument zu dem was er \textit{Virtuose War} nennt. Krieg, so der Derian, stellt in der Post-Moderne die Absicht dar, politischen und moralischen Wandel nötigenfalls durch den Einsatz von militärischer Gewalt hervorzurufen. Das \textit{jus bellum} Konzept, das an vielen Stellen den Einsatz militärischer Gewalt in Form von "humanitären Interventionen" rechtfertigt, misst nun der militärischen Gewaltanwendung gewisse inhärente Qualitäten zu, nämlich "Unschuldige" vor "Bösen" zu schützen sowie Demokratie und liberale Marktwirtschaft auszubreiten. \textit{Virtuouse War} verwandelt politische Konflikte in ethische Imperative die, wenn nötig, auch mit Waffengewalt durchgesetzt werden. Dabei ist eine weitere "Tugend" dieses Modus der Gewaltanwendung, die eignen Verluste auf ein absolutes Minimum zu beschränken und bestenfalls komplett zu vermeiden. Coker argumentiert hier, dass in modernen Konflikten wenig über die eignen Verluste nachgedacht wurde oder den Preis, in Form von Menschenleben, der für den Erfolg militärischer Operationen zu zahlen war als zu hoch angesehen hätten (\cite{coker_postmodern_2008}S. 8). 
Cokers Argument zielt hier vermutlich darauf ab, dass Verluste in post-modernen Kriegen nicht mehr nur aus taktischen Erwägungen vermieden werden, sondern dass das vermeiden eigener Verluste zu einem moralischen Imperativ geworden ist. Als Idealtyp eines solchen Krieges gilt dabei der Kosovo Konflikt (\cite{DerDerian2001} S. xv ff.; S. 201).\\
Dabei wird das Bild eines sauberen und blutleeren Krieges zu vermitteln. In post-moderne Kriegen, wird häufig versucht beispielsweise durch den Einsatz von Bildern aus den Boardkameras von "Präzisionsmunition" den Eindruck zu vermitteln, dass sich die humanitären Ziele die die Gewaltanwendung legitimiert haben, auch im Konflikt selber wiederfinden lassen. Besonders deutlich wird dies in einem Vergleich zwischen dem Vietnam Krieg und post-modernen, \textit{virtuous wars} in der post-Vietnam Ära. Waren in Vietnam noch sogenannte \textit{Body Counts}, also das zählen getöteter vermeintlicher Feinde ein Indikator für den Erfolg einer militärischen Operation wurde in den Folgeenden Konflikten von der Veröffentlichung solcher Zahlen weitgehend abgesehen \cite{DerDerian2001} S. xv ff.; siehe auch: \cite{graham_enemy_2005}).\\
Festhalten lässt sich hier daher, dass post-moderne Kriege durch ein \textit{jus bellum} Konzept legitimiert werden, das Bild eines sauberen Krieges vermitteln sollen und eigne Verluste auf ein Minimum beschränkt werden.
\subsection{Post-Moderner Krieg und RMA}
Wie oben dargelegt zeichnen sich post-moderne Kriege dadurch aus, dass sie nach \textit{jus bellum} Kriterien geführt und legitimiert werden, keine eignen Verluste hervorrufen sowie, zumindest an der Oberfläche, möglichst sauber und blutleer wirken.\\
Hier beginnt der zweite Teil von der Derians Argument. Um die oben beschriebenen Vorgaben zu erreichen wird die Gewaltanwendung nach Möglichkeit aus der Distanz heraus betrieben - durch den Einsatz von "Präzisionsmunition", bemannten und unbemannten Flugzeugen und sogenannten "Cruise Missiles". Auch hier ist der Idealtyp, soweit dies bis jetzt beurteilt werden kann, wieder der Kosovo Konflikt. Eine humanitäre Intervention, ausschließlich aus der Luft geführt, ohne jede eigne Verluste und mit einem Meta-Narrativ das versucht ausschließlich Bilder einen präzisen Krieges zu zeigen in dem nur getroffen wird was getroffen werden soll ({\cite{DerDerian2001} S. xv ff.})
\begin{quote}
"We hit what we were aiming for, [...]\\
 But we did not mean to hit the Chinese Embassy." (anonymer NATO Sprecher, zitiert in:\cite{gordon_crisis_1999}) 
\end{quote}
Die Entwicklung des Krieges hin zur post-moderne geht nicht nur mit einer Weiterentwicklung in der (Militär) Technologie sondern auch mit dem was als \textit{Revolution in military affairs} (\textit{RMA}) bezeichnet wird einher. Als RMA wird eine Zäsur in der militärischen Entwicklung bezeichnet, bei der nicht nur neue Technologien Einzug halten, sondern diese auch einen tiefgreifenden Wandel innerhalb der Institutionskultur und der militärischen Taktik hervorrufen.\footnote{\cite{singer_wired_2010} S.179 ff.} \\
Die aktuelle \textit{RMA} beschreibt vor allem das auftauchen komplexer, vernetzter Systeme innerhalb des Militärs. Wenngleich der Einsatz von Computer und Netzwerk Technologie im Militär sicherlich kein neues Phänomen ist, hat sich der die Qualität der Netzwerke, die Quantität ihrer Verbreitung und die Inkorporation in den institutionellen Modus des Militärs soweit gesteigert, das mittlerweile von einer \textit{RMA} gesprochen werden kann.
Zentral ist in der aktuellen Diskussion die Beiträge des US-Navy Angehörigen Arthur Chebrowski. Chebrowski argumentiert, dass der jüngste Sprung in der Netzwerk-Technologie zu einem deutlich verändertem Modus führt, in dem Informationen und Software wichtiger werden als Veränderungen der Hardware und damit der eigentlichen Waffensysteme (\cite{singer_wired_2010}S.179 ff.).
Netzwerke ermöglichen dabei eine sehr schnelle Weiterleitung von Informationen und Befehlen auch über weite Distanzen.
Dies hat, verbunden mit weiteren Entwicklungen in der Waffentechnologie, beispielsweise der gesteigerten Präzision und Reichweite von konventionellen Waffen wie z.B. der sogenannten "Smart-Bombs", die für sich genommen aber keine \textit{RMA} darstellen (vgl. an dieser Stelle z.B. \cite{ignatieff_virtual_2000}), fundamentale Auswirkungen auf den Modus der Kriegsführung. 
Bei Der Derian aber auch bei Ignatieff oder Singer werden dabei immer wieder zwei Teilbereiche angesprochen in denen sich diese Auswirkungen besonders Deutlich zeigen: Sichtbarkeit und Distanz (\cite{DerDerian2001}, \cite{ignatieff_virtual_2000}, \cite{singer_wired_2010}). 
Zum einen verspricht die neue Technik einen umfassenden Überblick über die Gegebenheiten des Schlachtfeldes und darüber hinaus. 
Netzwerke sollen es hier möglich machen, Informationen auch von unbemannten Systemen wie z.B. "Drohnen" schnell zu übermitteln. 
Der Derian schreibt zu diesem Thema das die neue Technologie eine umfassende Wissenslage verspricht, die das Clausewitzsche Paradigma vom "\textit{Nebel des Krieges}" scheinbar in Frage stellt (\cite{DerDerian2001} S.1 ff.).
Das diese Entwicklung dabei keineswegs dem Militär Vorbehalten ist, macht Coker bereits deutlich, in seiner Argumentation über das fraktierte Narrativ des Krieges und die verbesserten Möglichkeiten von Medien (\cite{coker_collision_2012} S.4 f.). \\
Zum anderen können durch Netzwerke Informationen auch schneller über große Distanzen übermittelt werden. 
Dies, in Kombination mit der vermeintlich verbesserten Sichtbarkeit des Schlachtfeldes, hat zur Folge, dass militärische Gewalt über lange Distanzen eingesetzt werden kann. 
Beispiele lassen sich dabei in den meisten post-modernen Konflikten finden, an denen Industrienationen beteiligt sind und umfassen beispielsweise "Cruise Missiles", den Einsatz von "Luftschlägen" oder den Einsatz von bewaffneten "Drohnen". 
Das Vertrauen in die Technik ermöglicht es hier, quasi aus sicherer Entfernung das Schlachtfeld als vollständig Sichtbar zu betrachten und Gewalt weitestgehend ohne Eigengefährdung einzusetzen (\cite{DerDerian2001} S.1 ff).
In diesem Szenario, so Der Derian, ermöglicht das Vertrauen in die Technik, das durchsetzen ethischer Imperative mit Waffengewalt, da die Militärtechnik anscheinend in der Lage ist, ohne Gefährdung der eignen SoldatInnen \textit{jus bellum} Kriterien, insbesondere denen des \textit{jus in bello} militärische Gewalt anzuwenden. 
Die Technik ermöglicht somit das politische Fragen zu ethischen Imperativen erklärt werden können und durch den Einsatz der Technik militärisch Durchsetzbar erscheinen. 
Der Derian spricht hier von einem "\textit{technological fix}" für politische Fragestellungen der sich, zusammen mit dem Vertrauen in die Wirkmächtigkeit der Technik, das er als "\textit{Techo-Fetischismus}" bezeichnet zu einer Rückkopplungschleife verbindet in der für jede noch nicht Durchsetzbare politische Forderung nur der passende "\textit{technological fix}" gefunden werden muss, um diese militärisch Durchsetzbar zu machen, da der Technik eine prinzipiell eine fast unbeschränkte Wirkmächtig zugeschrieben wird (\cite{DerDerian2001}S. 199 ff.).
\end{document}
