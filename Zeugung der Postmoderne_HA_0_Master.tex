\documentclass[11pt,a4paper,oneside,numbers=noenddot,bibliography=totocnumbered,DIV=13]{scrartcl}
%\usepackage[lmargin=2cm,rmargin=4cm,tmargin=3cm,bmargin=3cm]{geometry}
%Sprache
\usepackage[utf8]{inputenc}
\usepackage[T1]{fontenc}
\usepackage[ngerman]{babel}
%Typografie
\usepackage[bitstream-charter]{mathdesign}
%\usepackage{txfonts}
\usepackage[activate={true,nocompatibility},final,tracking=true,kerning=true,spacing=true,factor=1100,stretch=10,shrink=10]{microtype}
\usepackage{url}
%überschriften in Helvetica
\usepackage{sectsty}
\allsectionsfont{\fontfamily{phv}\selectfont\normalsize} 
%Schusterjungen
\clubpenalty = 10000 
\widowpenalty = 10000 
\displaywidowpenalty = 10000
% Zeilenabstand
\usepackage{setspace}
\onehalfspacing
%Zitate
\newcommand{\block}[2]{{\footnotesize\singlespacing\blockquote{\enquote{#1}{#2}}}\vspace{+2mm}}% <-- Hier kannst du den Abstand veraendern
\usepackage{tocstyle} 
%\usetocstyle{KOMAlike}%Find ich eigentlich gut 
\addtocontents{toc}{\protect\fontfamily{phv}} 


\setkomafont{title}{\normalfont\bfseries}


\begin{document}

\titlehead{Freie Universität Berlin \hfill Autor: Jan Opper\\
Otto-Suhr Institut für Politikwissenschaft\hfill Mat.-Nr.: 4655381\\
SE: Geburt der Moderne oder Zeugung der Post-Moderne\hfill {E-Mail: Opperj@zedat.fu-berlin.de}\\
Leitung: Gerd Harders\hfill 15.10.2014\\
Sommersemester 2014\\} 
\title{\large Tod und Technik} 
\subtitle{\small DSigmund Freud und der Post-Moderne Krieg\}}
\maketitle
\newpage
\tableofcontents
\newpage
\section{Test}
Test




\newpage
\bibliography{Postmoderne_HA_Bib}
\end{document}
