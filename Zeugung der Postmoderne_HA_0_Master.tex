\documentclass[11pt,a4paper,oneside,numbers=noenddot,bibliography=totocnumbered,DIV=13]{scrartcl}
%\usepackage[lmargin=2cm,rmargin=4cm,tmargin=3cm,bmargin=3cm]{geometry}
%Sprache
\usepackage[utf8]{inputenc}
\usepackage[T1]{fontenc}
\usepackage[ngerman]{babel}
%Typografie
\usepackage[bitstream-charter]{mathdesign}
%\usepackage{txfonts}
\usepackage[activate={true,nocompatibility},final,tracking=true,kerning=true,spacing=true,factor=1100,stretch=10,shrink=10]{microtype}
\usepackage{url}
%ueberschriften in Helvetica
%\usepackage{sectsty}
%\allsectionsfont{\fontfamily{phv}\selectfont\normalsize} 
%Schusterjungen
\clubpenalty = 10000 
\widowpenalty = 10000 
\displaywidowpenalty = 10000
% Zeilenabstand
\usepackage{setspace}
\onehalfspacing
%Zitate
\newcommand{\block}[2]{{\footnotesize\singlespacing\blockquote{\enquote{#1}{#2}}}\vspace{+2mm}}% <-- Hier kannst du den Abstand veraendern
\usepackage{tocstyle} 
%\usetocstyle{KOMAlike}%Find ich eigentlich gut 
\addtocontents{toc}{\protect\fontfamily{phv}} 


\setkomafont{title}{\normalfont\bfseries}


\begin{document}

\titlehead{Freie Universität Berlin \hfill Autor: Jan Opper\\
Otto-Suhr Institut für Politikwissenschaft\hfill Mat.-Nr.: 4655381\\
SE: Geburt der Moderne oder Zeugung der Post-Moderne\hfill {E-Mail: Opperj@zedat.fu-berlin.de}\\
Leitung: Gerd Harders\hfill 15.10.2014\\
Sommersemester 2014\\} 
\title{\large Tod und Technik} 
\subtitle{\small DSigmund Freud und der Post-Moderne Krieg\}}
\maketitle
\newpage
\tableofcontents
\newpage
\section{Einleitung}
Mit dem \textit{Golfkrieg} 1991\footnote{der Terminus meint die Intervention einer Koalition verschiedener Staaten 1991 in Kuwait und den Irak unter Führung der USA} begann in der einschlägigen Literatur die Diskussion über eine Veränderte Art und Weise der Kriegsführung industrieller Staaten. Spätestens seit der NATO Intervention in das Kosovo wurde in weiten Teilen der Literatur dabei von einem neuen Modus (westlicher) Kriegsführung gesprochen. Besondere Beachtung fanden dabei die veränderte Strategie, gepaart mit dem Einsatz neuer Technik wie beispielsweise dem Einsatz von sogenannten \textit{Präzisionswaffen} oder \textit{smart-bombs} aus großer Distanz wie zum Beispiel der Luft oder von Schiffen und den damit verbundenen ungleichen Verlustzahlen zwischen den Kriegsparteien, sowie der Repräsentation der Kriege in den Medien. 
\newpage
\section{Post-Moderner Krieg}
In diesem Abschnitt soll näher auf die Entwicklung vom modernen zum post-modernen Krieg sowie auf die Charakteristika des post-modernen Krieges   eingegangen werden. Einen wichtigen Ausgangspunkt zur Beschreibung dessen was in diesem Text als post-modernen Krieg verstanden wird, bietet James der Derians Konzept des \textit{Virtous War}. Der Derian argumentiert das Kriege industrieller Staaten in der Post-Vietnam, Post-Kalter Krieg Ära zunehmend von zwei Kennzeichen geprägt sind: Zum einen werden post-moderne Kriege als \textit{jus bellum} dargestellt - als gerechter Krieg und notwendigen Einsatz militärischer Gewalt zum Schutz von zivilen Opfern. Zum anderen wird zum erreichen der militärischen Ziele ein \textit{technological fix} angewendet. In post-modernen Kriegen findet sich der Einsatz neuer Technologie um einen vermeintlich sauberen Krieg zu ermöglichen, in dem kaum eigne Verluste sowie zivile Tote zu beklagen sind. Beide Charakteristika haben die Konzept der (post-)modernen Kriegsführung nachhaltig Verändert.(\cite{DerDerian2001}) \\
Dabei gilt es zu beachten, das die post-moderne Kriege selten in ihrer Reinform auftauchen, ein Prototyp findet sich möglicherweise in der NATO Intervention gegen Serbien von März bis Juni 1999. Allerdings finden sich in den meisten Fällen Mischformen, in denen die Charakteristika die hier dem post-modernen Krieg zugeschrieben werden, nur sehr begrenzt zum tragen kommen. Der post-moderne Krieg muss daher eher als Entwicklung hin zu etwas, als Tendenz denn als alleinige Realität der Kriegsführung verstanden werden. Des weiteren ist der hier beschriebene \textit{Virtous War} eher in der Kriegsführung industrieller bzw. post-industrieller Staaten zu finden, da die Möglichkeit einen \textit{technological fix} einzuführen, (noch) stark von den ökonomischen Möglichkeiten eines Staates abhängig ist. Zwar finden sich Stimmen, die argumentieren, das auch Kriege bspw. zwischen bewaffnetet Gruppen zunehmend die Form eines \textit{Virtuose War} annehmen, allerdings finden sich doch noch signifikante unterscheide zwischen den technischen Möglichkeiten westlicher Industriestaaten und nicht-staatlichen Akteuren in der Kriegsführung.\\ 
\subsection{Vom modernen zum post-modernen Krieg}
%Post-Moderne als Intellektuelle Strömung
Christopher Coker sieht in der um die 1910/1920 entstehenden Anfängen der post-modernen Strömung, eine intellektuelle Unzufriedenheit mit der von der Moderne vertretenen Vorstellung einer klar geordneten, wissenschaftlich zugänglichen und widerspruchsfreien Welt. Die Post-Moderne als intellektuelle Strömung, begreift die Welt als fragmentiert, uneindeutig und widersprüchlich. Gleichzeitig wird in der Post-Moderne den Subjekten dieser Welt auch eine größere \textit{agency} zugeschrieben als dies in der Moderne der Fall wahr.(\cite{coker_collision_2012} S.4)\\
Bezogen auf den post-modernen Krieg argumentiert Coker, dass dieser sich sich ebenfalls durch eine Auflösung eines großen Meta-Narratives auszeichnet, das während moderner Kriege dazu benutzt werden konnte die Bevölkerung zu mobilisieren.\footnote{ebd.}\\
Edward Luttenwak argumentiert in seinem viel beachteten Artikel \textit{A post-Heroic Military Policy}, dass vor dem kalten Krieg, Kriegseintritte zumeist mit einer schnellen Mobilmachung der Bevölkerung einhergingen, während nach Beendigung der Feinseligkeiten eine ebenso schnelle De-Mobilisierung erfolgte, die das Militär auf den prä-Kriegs Zustand zurück brachte. Luttenwak argumentiert weiter, dass sich diese im Kalten Krieg geändert hat, da hier eine ständige Bereitschaft an starken militärischen Kräften erforderlich war, um eine wirksame Abschreckung zu gewährleisten. Allerdings sind insbesondere nach Ende des Vietnamkrieges, aufgrund von politischen Beschränkungen, nur wenige dieser Kräfte für tatsächliche Kampfhandlungen, jenseits der direkten Landesverteidigung einsetzbar.(\cite{Luttwak1996} S. 34ff.)\\
Daher stellt der Vietnamkrieg eine Zäsur in der Entwicklung von der modernen zur post-modernen Kriegsführung dar. Während die Streitkräfte der USA in der eignen Bevölkerung nach dem Sieg über die Achsenmächte und dem Ende des zweiten Weltkrieges enorme Popularität und Unterstützung genossen, wandelte sich dieses Bild während des Vietnamkrieges. Die hohen Verluste sowie der Rückzug der US-Streitkräfte aus Vietnam wurde in der Bevölkerung und auch im Militär selber als Niederlage wahrgenommen. Dies führte zu einer Phase der Reflexion und des Umdenkens sowohl im Militär als auch in der US-Amerikanischen Bevölkerung.[\cite{olson_war_2007} S.563 f.]\\
integraler Bestandteil der \textit{Containment} Strategie, als Prinzip der  amerikanischer Außenpolitik, war es die Ausbreitung des Einflussbereichs "fremder Mächte", speziell der Sowjetunion und anderer "sozialistischer"\footnote{"sozialistischer" meint hier in Konkurrenz zu der im Westen dominanten Kapitalismusform stehend; Ziel ist nicht eine Debatte darüber anzuregen ob und in wie weit das polit--ökonomische System der Sowjetunion, Chinas und anderer tatsächlich einem sozialistischen Gedanken entspricht} Staaten einzudämmen ohne dabei tatsächlich eine militärische Auseinandersetzung größeren Ausmaßes mit den betreffenden Staaten führen zu müssen.(\cite{olson_war_2007}S.566 ff.)\\         
Mit dem Scheitern der Strategie, bewaffnete Konflikte zu vermeiden und unter dem Eindruck von massiven eignen Verlusten, auf "Nebenkriegsschauplätzen", jenseits einer eigentlichen Konfrontation mit der Sowjetunion, geriet auch die \textit{Containment} Doktrin zunehmend unter Druck. Es kam erstmals seit Beginn des Kalten Krieges zu einer breiten Debatte um Außenpolitik.\footnote{ebd.}\\
Mit dem Ende des Kalten Krieges und der Zunahme von sogenannten \textit{Operations other than War}, wie Humanitären Interventioen oder Peacekeeping Einsätzen schon vor Ende des Kalten Krieges befand sich der Konsens bezüglich der \textit{Containment} Strategie der USA endgültig in Auflösung.(\cite{coker_postmodern_2008}S.7 f.)\\
Luttwaks Argument setzt genau an dieser Stelle an. Während, wie bereits erwähnt, im Kalten Krieg massive Anstrengungen unternommen wurden um ein quantitativ starkes Militär als wirksame Abschreckung gegen bewaffnete Konflikte unterhalb der Schwelle eines Nuklearkriegs zu haben, sind eben jene personenstarken Teilbereiche der Streitkräfte besonderen ineffektiv für den tatsächlichen Einsatz, da sie in Kampfhandlungen häufig auf kurze Distanz eingesetzt werden müssen und somit Gefahr laufen hohe eigne Verluste zu erleiden. Luttwak weist darauf hin das, insbesondere in Demokratien, hohe eigne Verluste meist zu einer errosionsrtigen Abnahme der öffentlichen Unterstützung für militärische Einsätze nach sich ziehen. (\cite{Luttwak1996}S.35 f.)\\
Ausdruck eben jener Furcht vor dem Verlust der öffentlichen Unterstützung bei zu hohen eigenen Verlusten stellt die sogenannte \textit{Weinberger Doctrine} (auch \textit{Weinberger-Powell} dar. Diese 1984 Doktrin zum Einsatz von militärischen Kräften durch die USA, reflektiert dabei die Erfahrungen aus Vietnam, aber auch die hohen Verluste der USA während des Peacekeeping Einsatzes der \textit{Multinational Force in Lebanon} 1982-1984 in Beirut. Weinberger fordert den Einsatz von Streitkräften auf ein minumum zu begrenzen und diese nur einzusetzen wenn sich die Regierung der öffentliche Unterstützung sicher sein könne.(\cite{Weinberger28.11.1984})  Die Weinberger Doktrin nimmt auch eine Beobachtung vorweg, die sich später sowohl in Cokers Charakterisierung des post-modernen Krieges als auch in Der Derians Beschreibung des \textit{Virtuous War} finden lässt - die zunehmende Durchmischung von Frieden und Krieg. Coker nennt es Ironie, dass während des Peacekeeping Einsatzes im Libanon mehr US Soldaten starben als während mancher Kampfhandlung (\cite{coker_postmodern_2008}S.7 ff.), während Der Derian eher auf den Aspekt der mit militärischen Drohungen garnierten Diplomatie hinweist.(\cite{DerDerian2001}S.198)\\
% Übergang verbessern
Das Ende des Kalten Krieges stellte dabei nicht nur den Beginn einer neuen militärisch/aussenpolitischen Ära da, sondern wurde in den USA auch als ein Übergang von einer Bi- zu einer Multipolaren Weltordnung bewertet, in der die bisherigen Strategien zur Abschreckung eines potentiellen Gegners, nur begrenzt geigten sind. Gleichzeitig  kam es unter der Clinton-Administration zu einer endgültigen Abkehr von der, inzwischen Überholten, \textit{Containment} Strategie und der Einführung einer idealistischeren Außenpolitik. Das Konzept des \textit{Enlargement}, das auf eine Ausbreitung von liberaler Marktwirtschaft und Demokratie setzt und dabei auch eher geneigt ist militärische Interventionen nach dem Prinzip des \textit{jus bellum} im Sinne einer Humanitären Intervention einzusetzen. (\cite{DerDerian2001}S.17 ff.)\\ Gerade das Prinzip des \textit{jus bellum} spielt hier eine große Rolle da es den Einsatz militärischer Gewalt legitimieren kann, indem es spezifische Bedingungen definiert, unter denen der Einsatz von Gewalt gerechtfertigt ist (vgl. beispielsweise:\cite{rengger_just_2002}) So bemerkt beispielsweise Banta in seinem Artikel, dass seit dem Ende des Kalten Krieges keine liberale Demokratie militärische Gewalt eingesetzt hat, ohne sich zumindest teilweise auf eine humanitäre Begründung zu stützen (\cite{banta_virtuous_2011}S.280 ff.).    
\subsection{Charakteristika des Post-Modernen Kriegs}
Ausgangspunkt der Betrachtung der Charakteristika des post-modernen Krieges ist die im vorherigen Abschnitt dargelegte veränderte Situation nach dem Ende des Kalten Krieges. Insbesondere die Einführung des \textit{Enlargements} Konzepts führte dabei zu einem stärkeren Einsatz militärischer Gewalt, zumeist in Verbindung mit sogenannten \textit{Operations other than War} wie "humanitären Intervention" oder Peackeeping-Einsätzen. Im Konzept des \textit{Enlargments} angelegt ist auch eine Legitimation dieser Gewaltanwendung zum Zweck der Ausbreitung von Konzepten wie Demokratie oder dem Schutz von Menschenrechten, innerhalb moderner \textit{jus bellum} Konzepte wie der \textit{Responsibility to Protect}.\\
Das \textit{jus bellum} Konzept ist dabei ein zentraler Punkt für den post-modernen Krieg westlicher Industrienationen. Nach der Auflösung der \textit{Containment} Strategie nah dem Wegfall der Sowjetunion als primäre Bedrohung wurde auch eine auf starken konventionellen und nuklear-Streitkräften beruhende Abschreckungsdoktrin mit der Vorbereitung auf größere Konflikte mit dem Gegner, beispielsweise die Planungen für eine Militärische Auseinandersetzung am sogenannten "Fulda Gap"(\cite{olson_war_2007}S. 566) zur Eindämmung des gegnerischen Machtbereichs,zunehmend nutzlos. Dagegen findet sich im \textit{Enlargement} die tatsächliche Anwendung militärischer Gewalt gegen verschiedene potentielle \textit{states of concern} legitimiert durch das \textit{jus bellum} Konzept (\cite{DerDerian2001}S. I ff.). Hier findet sich auch eine klare Abkehr von der in \textit{Weinberger Doktrin} vertretenen Position das der Einsatz von US Truppen auf ein Minimum beschränkt werden soll, stattessen wird zunehmend das Konzept eine \textit{liberalen Interventionismus} betrieben (\cite{lafeber_rise_2009}).  
%jus Bellum
Das \textit{jus bellum} Konzept stellt eine Denk-Tradition dar, die ihren beginn im Aufkommen des Christentums und des damit Verbundenen früh-christlichen Radikalpazifismus hat. In dieser Zeit, stellte \textit{jus bellum} eine Möglichkeit dar, dieses Gewaltverbot zu umgehen indem es die Anwendung militärischer Gewalt unter bestimmten Bedingungen, als Vereinbar mit dem christlichen Glauben ansah. \textit{Jus bellum} war zwar schon in seinen Anfängen ein säkulares Konzept, allerdings hat sich seit dem 17 Jhrd. die Debatte weiter ausdifferenziert und liegt und sich von seiner Gestalt als ethisch-Moralische Handlungsmaxime gelöst, so das es heute vor allem in seiner verrechtlichen Form, beispielsweise in Artikel 7 der UN Charter oder im internationalem Kriegsvölkerrecht vorliegt. Aktuell findet sich das \textit{jus bellum} Konzept auch in der vor einigen Jahren diskutierten \textit{Responsibility to Protect} (für \textit{jus bellum} vgl.: \cite{2002} S. 926 ff.; siehe auch: \cite{bellamy_responsibilities_2008} für \textit{Responsibility to Protect} siehe: \cite{international_commission_on_intervention_and_state_sovereignty_responsibility_2001}).\\
Eine neuere Debatte um das \textit{jus bellum} entstand im Verlaufe der 60er, 70er und 80er vor dem Hintergrund des Vietnam Krieges. Johnson argumentiert das während dieser Debatte primär ein Kriterienkatalog herangezogen wurde um zu prüfen, ob der Einsatz militärischer Gewalt im spezifischen Fall legitim ist bzw. war (\cite{johnson_contemporary_2013} S. 31 ff)).
Dazu werden zumeist zwei Bereiche des \textit{jus bellum} Konzept betrachtet: Einerseits das \textit{jus ad bellum} und andererseits das \textit{jus in bello}. 
\begin{quote}
“[…] beginning with the classic criteria of sovereign authority, just cause, and right intention, including the end of peace; then adding the prudential criteria widely applied today—reasonable hope of success, proportionality of ends, and lack of reasonable alternatives (last resort); and finally defining right conduct in war in terms of discrimination and proportionality of means.” (\cite{johnson_contemporary_2013}S. 32) \\
\end{quote}
\textit{Jus ad bellum} meint daher die Kriterien die vor dem Einsatz militärischer Gewalt gegeben sein müssen um diese als legitim zu sehen. Hierbei handelt es sich vor allem um den \textit{just cause} welcher vor allem den Schutz von "Unschuldigen" vor "Bösem", aus altruistischen Gründen, mit der Gewaltanwendung als \textit{last resort} und dem Ziel einen Frieden nach dem Ende der Gewaltanwendung wieder herzustellen.
 \textit{Jus in bello} bewertet das Verhalten der intervenierenden Seite im Konflikt selber und spiegelt sich auch beispielsweise im Kriegsvölkerrecht wieder. Dies heist vor allem der Schutz von Zivilisten und die Verhältnismäßigkeit bei Einsatz von Gewalt (\cite{rengger_just_2002}S.358 ff.). Aktuell ist vor allem eine Debatte, ob das Konzept um einen weiteren Teilbereich ergänzt werden muss, der das Verhalten nach Ende der Gewaltanwendung regelt. Die Notwendigkeit eines \textit{jus post bellum} ist allerdings umstritten (\cite{bellamy_responsibilities_2008}; siehe auch: \cite{banta_virtuous_2011}).\\    
%Sichtbarkeit des Schlachtfeldes
Hinzu kommt, dass so Christopher Coker,das neue Medien, allen voran das Fernsehen und das Internet auch für eine neue Sichtbarkeit des Krieges stoßen. Insbesondere zivile Opfer werden durch neue Medien besser sichtbar und Zwischenfälle können immer wieder Debatten über die Legitimität der militärischen Gewaltanwendung auslösen. Während sich dieser Trend schon zur Zeit des Vietnamkrieges feststellen lässt, hat er mit der breiten Einführung des Internets nochmal zugenommen. Eigene und zivile Verluste werden besser sichtbar und kommen, zumindest theoretisch, ohne die \textit{Gatekeeper-Funktion} etablierter Medien aus.\\
Diesbezüglich geht Cocker auch von einem einer Auflösung eine in sich selbst begründeten Narratives des Krieges aus und setzt an seine stelle viele fraktionaliserte Sichtweisen (\cite{coker_collision_2012} S.4 f.):
\begin{quote}
The image of civilians recovering from a mistimed airstrike; the image of children in the
rubble of a village—all diminish reputations by exposing the illegitimacy of power, or its use. War may still be
necessary but it is now no longer redeeming. (\cite{coker_collision_2012}S.5) 
\end{quote}  
Während Coker daher eine bessere Sichtbarkeit des Schlachtfeldes durch neue Medien feststellt, stellt er gleichzeitig fest das es seit dem Vietnam Krieg auch eine Gegenbewegung, innerhalb aber auch außerhalb des Militärs gibt, um die zerstörerische Natur des Krieges zu verbergen. Dies geschieht, so Coker, vor allem durch die Verwendung von neuen Sprachregelungen wie beispielsweise \textit{Colleteral Damage} (\cite{coker_postmodern_2008}S. 8 f.)\\  
%Der Derians Virtouse War
Vor dem Hintergrund der oben beschriebenen Veränderungen entwickelt auch James der Derian eine Hälfte seines Argument zu dem was er \textit{Virtuose War} nennt. Krieg, so der Derian, stellt in der Post-Moderne die Absicht dar, politischen und moralischen Wandel nötigenfalls durch den Einsatz von militärischer Gewalt hervorzurufen. Das \textit{jus bellum} Konzept, das an vielen Stellen den Einsatz militärischer Gewalt in Form von "humanitären Interventionen" rechtfertigt, misst nun der militärischen Gewaltanwendung gewisse inhärente Qualitäten zu, nämlich "Unschuldige" vor "Bösen" zu schützen sowie Demokratie und liberale Marktwirtschaft auszubreiten. \textit{Virtuouse War} verwandelt politische Konflikte in ethische Imperative die, wenn nötig, auch mit Waffengewalt durchgesetzt werden. Dabei ist eine weitere "Tugend" dieses Modus der Gewaltanwendung, die eignen Verluste auf ein absolutes Minimum zu beschränken und bestenfalls komplett zu vermeiden. Coker argumentiert hier, dass in modernen Konflikten wenig über die eignen Verluste nachgedacht wurde oder den Preis, in Form von Menschenleben, der für den Erfolg militärischer Operationen zu zahlen war als zu hoch angesehen hätten (\cite{coker_postmodern_2008}S. 8). 
Cokers Argument zielt hier vermutlich darauf ab, dass Verluste in post-modernen Kriegen nicht mehr nur aus taktischen Erwägungen vermieden werden, sondern dass das vermeiden eigener Verluste zu einem moralischen Imperativ geworden ist. Als Idealtyp eines solchen Krieges gilt dabei der Kosovo Konflikt (\cite{DerDerian2001} S. xv ff.; S. 201).\\
Dabei wird das Bild eines sauberen und blutleeren Krieges zu vermitteln. In post-moderne Kriegen, wird häufig versucht beispielsweise durch den Einsatz von Bildern aus den Boardkameras von "Präzisionsmunition" den Eindruck zu vermitteln, dass sich die humanitären Ziele die die Gewaltanwendung legitimiert haben, auch im Konflikt selber wiederfinden lassen. Besonders deutlich wird dies in einem Vergleich zwischen dem Vietnam Krieg und post-modernen, \textit{virtuous wars} in der post-Vietnam Ära. Waren in Vietnam noch sogenannte \textit{Body Counts}, also das zählen getöteter vermeintlicher Feinde ein Indikator für den Erfolg einer militärischen Operation wurde in den Folgeenden Konflikten von der Veröffentlichung solcher Zahlen weitgehend abgesehen \cite{DerDerian2001} S. xv ff.; siehe auch: \cite{graham_enemy_2005}).\\
Festhalten lässt sich hier daher, dass post-moderne Kriege durch ein \textit{jus bellum} Konzept legitimiert werden, das Bild eines sauberen Krieges vermitteln sollen und eigne Verluste auf ein Minimum beschränkt werden.
\subsection{Post-Moderner Krieg und RMA}
Wie oben dargelegt zeichnen sich post-moderne Kriege dadurch aus, dass sie nach \textit{jus bellum} Kriterien geführt und legitimiert werden, keine eignen Verluste hervorrufen sowie, zumindest an der Oberfläche, möglichst sauber und blutleer wirken.\\
Hier beginnt der zweite Teil von der Derians Argument. Um die oben beschriebenen Vorgaben zu erreichen wird die Gewaltanwendung nach Möglichkeit aus der Distanz heraus betrieben - durch den Einsatz von "Präzisionsmunition", bemannten und unbemannten Flugzeugen und sogenannten "Cruise Missiles". Auch hier ist der Idealtyp, soweit dies bis jetzt beurteilt werden kann, wieder der Kosovo Konflikt. Eine humanitäre Intervention, ausschließlich aus der Luft geführt, ohne jede eigne Verluste und mit einem Meta-Narrativ das versucht ausschließlich Bilder einen präzisen Krieges zu zeigen in dem nur getroffen wird was getroffen werden soll ({\cite{DerDerian2001} S. xv ff.})
\begin{quote}
"We hit what we were aiming for, [...]\\
 But we did not mean to hit the Chinese Embassy." (anonymer NATO Sprecher, zitiert in:\cite{gordon_crisis_1999}) 
\end{quote}
Die Entwicklung des Krieges hin zur post-moderne geht nicht nur mit einer Weiterentwicklung in der (Militär) Technologie sondern auch mit dem was als \textit{Revolution in military affairs} (\textit{RMA}) bezeichnet wird einher. Als RMA wird eine Zäsur in der militärischen Entwicklung bezeichnet, bei der nicht nur neue Technologien Einzug halten, sondern diese auch einen tiefgreifenden Wandel innerhalb der Institutionskultur und der militärischen Taktik hervorrufen.\footnote{\cite{singer_wired_2010} S.179 ff.} \\
Die aktuelle \textit{RMA} beschreibt vor allem das auftauchen komplexer, vernetzter Systeme innerhalb des Militärs. Wenngleich der Einsatz von Computer und Netzwerk Technologie im Militär sicherlich kein neues Phänomen ist, hat sich der die Qualität der Netzwerke, die Quantität ihrer Verbreitung und die Inkorporation in den institutionellen Modus des Militärs soweit gesteigert, das mittlerweile von einer \textit{RMA} gesprochen werden kann.
Zentral ist in der aktuellen Diskussion die Beiträge des US-Navy Angehörigen Arthur Chebrowski. Chebrowski argumentiert, dass der jüngste Sprung in der Netzwerk-Technologie zu einem deutlich verändertem Modus führt, in dem Informationen und Software wichtiger werden als Veränderungen der Hardware und damit der eigentlichen Waffensysteme (\cite{singer_wired_2010}S.179 ff.).
Netzwerke ermöglichen dabei eine sehr schnelle Weiterleitung von Informationen und Befehlen auch über weite Distanzen.
Dies hat, verbunden mit weiteren Entwicklungen in der Waffentechnologie, beispielsweise der gesteigerten Präzision und Reichweite von konventionellen Waffen wie z.B. der sogenannten "Smart-Bombs", die für sich genommen aber keine \textit{RMA} darstellen (vgl. an dieser Stelle z.B. \cite{ignatieff_virtual_2000}), fundamentale Auswirkungen auf den Modus der Kriegsführung. 
Bei Der Derian aber auch bei Ignatieff oder Singer werden dabei immer wieder zwei Teilbereiche angesprochen in denen sich diese Auswirkungen besonders Deutlich zeigen: Sichtbarkeit und Distanz (\cite{DerDerian2001}, \cite{ignatieff_virtual_2000}, \cite{singer_wired_2010}). 
Zum einen verspricht die neue Technik einen umfassenden Überblick über die Gegebenheiten des Schlachtfeldes und darüber hinaus. 
Netzwerke sollen es hier möglich machen, Informationen auch von unbemannten Systemen wie z.B. "Drohnen" schnell zu übermitteln. 
Der Derian schreibt zu diesem Thema das die neue Technologie eine umfassende Wissenslage verspricht, die das Clausewitzsche Paradigma vom "\textit{Nebel des Krieges}" scheinbar in Frage stellt (\cite{DerDerian2001} S.1 ff.).
Das diese Entwicklung dabei keineswegs dem Militär Vorbehalten ist, macht Coker bereits deutlich, in seiner Argumentation über das fraktierte Narrativ des Krieges und die verbesserten Möglichkeiten von Medien (\cite{coker_collision_2012} S.4 f.). \\
Zum anderen können durch Netzwerke Informationen auch schneller über große Distanzen übermittelt werden. 
Dies, in Kombination mit der vermeintlich verbesserten Sichtbarkeit des Schlachtfeldes, hat zur Folge, dass militärische Gewalt über lange Distanzen eingesetzt werden kann. 
Beispiele lassen sich dabei in den meisten post-modernen Konflikten finden, an denen Industrienationen beteiligt sind und umfassen beispielsweise "Cruise Missiles", den Einsatz von "Luftschlägen" oder den Einsatz von bewaffneten "Drohnen". 
Das Vertrauen in die Technik ermöglicht es hier, quasi aus sicherer Entfernung das Schlachtfeld als vollständig Sichtbar zu betrachten und Gewalt weitestgehend ohne Eigengefährdung einzusetzen (\cite{DerDerian2001} S.1 ff).
In diesem Szenario, so Der Derian, ermöglicht das Vertrauen in die Technik, das durchsetzen ethischer Imperative mit Waffengewalt, da die Militärtechnik anscheinend in der Lage ist, ohne Gefährdung der eignen SoldatInnen \textit{jus bellum} Kriterien, insbesondere denen des \textit{jus in bello} militärische Gewalt anzuwenden. 
Die Technik ermöglicht somit das politische Fragen zu ethischen Imperativen erklärt werden können und durch den Einsatz der Technik militärisch Durchsetzbar erscheinen. 
Der Derian spricht hier von einem "\textit{technological fix}" für politische Fragestellungen der sich, zusammen mit dem Vertrauen in die Wirkmächtigkeit der Technik, das er als "\textit{Techo-Fetischismus}" bezeichnet zu einer Rückkopplungschleife verbindet in der für jede noch nicht Durchsetzbare politische Forderung nur der passende "\textit{technological fix}" gefunden werden muss, um diese militärisch Durchsetzbar zu machen, da der Technik eine prinzipiell eine fast unbeschränkte Wirkmächtig zugeschrieben wird (\cite{DerDerian2001}S. 199 ff.).
\newpage
\section{Zwischenfazit Post-Moderner Krieg}
Im vorangegangenem Kapitel wurde ggezeigt, dass sich die Kriegsführung spätestens seit dem Ende des Kalten Krieges fundamental verändert hat. Dabei wird von breiten Teilen der Literatur angenommen, dass sich ein gänzlich  neuer Modus der Kriegsführung, vor allem in den westlichen und Industrienationen, etabliert hat. Dieser Modus  ist zwar keinesfalls alleinstehen, jedoch momentan, einer der am weitest Verbreiteten, wenn es zu Anwendung von Militärischer Gewalt durch \footnote{\cite{duyvesteyn_rethinking_2005}} westliche Staaten kommt. Die Literatur hat für diesen Modus unterschiedliche Namen geprägt, beispielweise "\textit{spectacle war}" oder "\textit{globalization war}"\footnote{\cite{bauman_wars_2001}} und schließen sich an eine Dabatte über "Neue Kriege" wie sie beispielweise von Kaldor oder Münkler\footnote{\cite{munkler_neuen_2004}} ausgelöst wurde an. Beide Autor\_innen argumentieren dabei ähnlich, nämlich das es in einem neuen Modus der Kriegsführung nicht primär um die Kontrolle von Territorium, sonder um das durchsetzen von politischen Forderungen, möglichst ohne eigne Verluste geht. Die hier verwendete Darstellung dieses Modus von James Der Derian\footnote{\cite{DerDerian2001}}  ist dabei eine der ausführlichsten, da er auf beide Punkte eingeht die diesen neuen Modus kennzeichne und im vorangegangenen Kapitel deutlich gemacht werden sollten.\\
Zum einen werden diese Neuen Kriege durch ethische Imperative begründet und durch die \textit{jus bellum} Doktrin legitimiert, zum anderen ist der Anspruch in diesen Kriegen keine eignen Verluste hinzunehmen, welcher in sich selbst zu einem weiteren ethischen Imperativ wird. Der Derian legt auch sehr gut da wie das was er einen "\textit{technological fix}" nennt angewendet wird um diese beiden, eigentlich gegensätzlichen Imperative zu verbinden. Dabei wird Technologie und neue strategisch/taktische Doktrinen eingesetzt um beiden Imperativen nachzukommen. Von Interesse für diese Arbeit ist dabei nicht primär ob der Einsatz neuer Technologien oder die Verbesserung bereits bestehender tatsächlich in der Lage ist die beiden Imperative zu verbinden, sondern das was von Der Derian mit dem Begriff \textit{technofetishism} beschrieben wird. Dies ist primär der Glaube und das Vertrauen in die Wirkmächtigkeit der Technik und die Möglichkeit in Konflikt stehende Impertaive zu verbinden bzw. die Verbindung durch die Weiterentwicklung der Technik möglich zu machen\footnote{\cite{DerDerian2001}S. 199 ff.}.      
\newpage
\section{Sigmund Freud}
Sigmund Freud gilt als Begründer der Psychoanalyse. 
Geboren 1856 in Wien beruht seine Psychoanalyse im wesentlichen auf drei Säulen: einer spezifischen Methodik, einer allgemeinen Metapsychologie sowie einer ebenfalls spezifischen Krankheitslehre. 
Als einer der Gründungstexte dessen was später die Psychoanalyse werden sollte gilt die von Freud in Zusammenarbeit mit Josef Breuer veröffentlichten Texte über die Behandlung von Bertha Pappenheim, besser Bekannt unter dem von Freud und Breuer verwendeten Pseudonym Anna O., durch das was sie \textit{Redekur} nannten.
Freuds große Leistung in der Ausarbeitung der Psychoanalyse, besteht dabei in der wissenschaftlichen Ergründung des Unbewussten (\cite{berkel_sigmund_2008}S. 13 f.).\\
Freud sah sich bei der Entwicklung seiner Theorie und Praxis als strenger Naturwissenschaftler und war skeptisch hinsichtlich allem was er als Vermischung von Philosophie und Naturwissenschaft sah, dennoch machte Freud selber und nach Freud auch andere Autoren die Erkenntnisse der Psychoanalyse nutzbar für die Sozialwissenschaften (\cite{lohmann_freud-handbuch:_2013}S. 10 ff.).
Für diese Arbeit sollen vor allem auf Freuds Theorie zur Kulturentwicklung Bezug genommen werden, insbesondere auf \textit{Zeitgemäßes über Krieg und Tod}, da Freud in ihnen die Psychoanalyse auf kollektive Prozesse verallgemeinert hat.        
\subsection{Freuds Kulturtheorie}
Freud entwickelt seine Kulturtheorie ausgehend von den Erkenntnissen der Psychoanalyse. Nach diesen ist das Leben der Menschen immer auf die Befriedigung ihrer triebhaften Bedürfnisse ausgerichtet. 
Allerdings erfolgt diese Triebbefriedigung nicht unmittelbar sondern wird mediert und reglementiert durch die Kultur. Freud geht dabei der Frage nach, wie die notwendigen vergesellschaftungsformen der Kultur, die Zusammenleben als sozialer Verband überhaupt erst möglich machen, in Konflikt stehen  mit dem was er als die (triebhafte) Natur des Menschen begreift (\cite{lohmann_freud-handbuch:_2013} S. 239). \\
Freud kommt bei der Verallgemeinerung seiner klinischen Ergebnisse fast zwangsläufig zu einer Kulturtheorie. 
Er verwendet dabei die Begriffe \textit{Kultur} und \textit{Zivilisation} weitgehend Synonym was zu einer breiten Kulturdefinition führt. 
Für Freud selbst legt dabei folgende Hulturdefinition vor: "all das worin sich das menschliche Leben über seine animalischen Bedingungen erhob"
% Kulturdefnition Freud mit Zitat belegen
Kultur ist daher die Instanz, die menschliches Zusammenleben über die unmittelbare Triebbefriedigung des Individuums hinaus gehen lässt. Freud versucht dabei in seinen Theorien die unbewusste Dimension dieser kollektiven und mit der unmittelbaren Triebentfaltung in Konflikt stehenden, kollektiven Prozesse sichtbar zu machen (\cite{berkel_sigmund_2008}S. 81 ff.)\\
\subsection{Der Mord als Beginn der sozialen Ordnung} 
Freuds kulturantropologische Schrift "Totem und Tabu" ist zwar nicht die erste Äußerung Freuds zur Kultur aber die für diese Arbeit bedeutsamste, da Freud in "Totem und Tabu" den Beginn der Kulturentwicklung nachzuvollziehen versucht ({\cite{berkel_sigmund_2008} S. 83 ff.). \\
Freud untersucht in seiner Schrift die Überscheidungen zwischen \glqq Neurotikern \grqq und nach dem Prinzip des Totemismus organisierte Gesellschaften von dem was er "Primitive" nennt. In beidem findet Freud \textit{Tabus} - starke und ambivalente Moralisch Ge- beziehungsweise Verbote, vor. Im Begriff \textit{Tabu} findet sich, laut Freud, das etwas heilig und Verboten zugleich ist. Er interpretiert das \textit{Tabu} daher als Ausdruck gegensätzlicher Regungen - einerseits des Begehrens und andererseits das Wissen um das verboten sein und die damit zusammenhängenden Schuldgefühle aus dem Begehren dessen was Verboten ist. Bei dem Vergleich mit "primitiven" Gesellschaften, beschäftigt sich Freud mit den zwei, für ihn, grundlegendsten \textit{Tabus} des Totemismus: Dem Inzestverbot und dem damit einhergehenden starken Exogamiegebot sowie dem Verbot das Totemtier ausshalb von rituellen Handlungen zu töten oder zu essen (\cite{berkel_sigmund_2008} S. 83 ff.).\\
Beide \textit{Tabus} sind für Freud grundlegend und beide sind für Freud auch ambivalent. Er argumentiert an mehren Stellen seiner Werke, das dass was nicht Begehrt wird auch nicht Verboten sein müsse. Nach seiner Sicht entspricht daher ein starkes moralisches Verbot auch einem ebenso starken, möglicherweise unbewussten, Verlangen (\cite{lohmann_freud-handbuch:_2013} S. 241 ff.).\\
Heutzutage erscheint es sinnig  die von Freud im Text aufgeworfenen Zusammenhänge als Denkbild und nicht als faktische Realität verstanden werden, da die meisten seiner Grundannahmen wie der Totemismus als vorherrschendes Organisationsprinzip "archaischer" Gesellschaften oder Freuds Vermutung das  kulturelle Ge- und Verbote genetisch von Genration zu Generation weitergegeben werden, als widerlegt gelten und im anbetrachtet zeitgenössischer Forschung nur schwer haltbar sind (\cite{lohmann_freud-handbuch:_2013} S. 169).\\
In seiner Argumentation setzt Freud vergleichbar mit dem Ödipuskomplex des Individuums den Vater analog zum Totemtier. Er geht dabei von einer \textit{Darwinistischen Urhorde} als die erste Gesellschaftsform aus. In dieser \textit{Urhorde} gibt es ein dominantes Männchen, den Vater, der alle Frauen der Horde für sich beansprucht und alle anderen männlichen Nachkommen oder mögliche Nebenbuhler vertreibt. Dem "Vater" kommt damit eine unangefochtene Vormachtstellung in der Gruppe zu. In Freuds Denkbild, taten sich die vertrieben Söhne eines Tages zusammen und erschlugen den übermächtigen Vater. Die Schuldgefühle bei der Tötung der geliebten und trotzdem verhassten Person es Vater, sowie die Vermeidung weiter, ähnlicher Konflikte, stellen für Freud die erste moralische Ordnung da (\cite{lohmann_freud-handbuch:_2013} S. 168 f.).\\
In der rituellen Tötung des Totemtieres und den damit verbundenen Sühneritualen wird, so Freud, die Tötung des Vaters immer wieder aufs neue begangen und gesühnt ohne das dies in der faktischen Tat mündet. Das Inzesttabu ergibt sich in diesem Bild aus der Vermeidung eine neue Vaterfigur zu erschaffen und so die im (rituellen) Vatermord verbundene Gesellschaft erneut zu spalten.  Insofern ist der Mord am Vater für Freud nicht nur auf Individueller sondern auch auf kollektiver Ebene bedeutsam und stellt den Beginn der Sittlichkeit dar (ebd.).\\
Freud argumentiert, dass sich in modernen Gesellschaften die beiden grundlegenden \textit{Tabus} weiter ausdifferenziert und zuletzt in eine komplexe soziale Ordnung übergegangen sind, letztendlich aber nach dem gleichen Schema wie die grundlegenden \textit{Tabus} funktionieren. Die Schuldgefühle im Angesicht des Vatermordes erklären für Freud dann auch die Ambivalenz von \textit{Tabus}, seien diese in Form von verrechtlicher oder moralischer sozialer Ordnung, in der tabuisierte Dinge zwar qua Trieb begehrt werden, sich aber auch Schuldgefühle ob des Begehrens dieser Dinge entwickeln. Die soziale Ordnung die Triebentfaltung des Individuums zwar unterdrückt, ermöglicht so erst soziales Zusammenleben indem sie verhindert das die Individuen ihren Trieben ungehemmt nachgehen (\cite{berkel_sigmund_2008} S. 90 f.).     
\subsection{Freud, Krieg und Tod}
Im vorherigen Kapitel wurde der Vatermord an den Anfang der Kulturentwicklung gesetzt. In Freuds Äußerungen zum Krieg wird vor allem das Thema \textit{Ambivalenz} erneut aufgegriffen, die Schulgefühle im Angesicht des Vatermordes aber zumeist durch die Unnvorstellbarkeit des eigenen Todes ersetzt.\\
Explizite Äußerungen Freuds zum Thema Krieg als kulturelles Ereignis finden sich vor allem in "Zeitgemäßes über Krieg und Tod" und im Briefwechsel mit Albert Einstein "Warum Krieg". In dieser Arbeit soll sich vor allem auf "Zeitgemäßes über Krieg und Tod" bezogen werden, da diese Quelle die Ergiebigste zu sein scheint.\\
"Zeitgemäßes über Krieg und Tod"\footnote{im weiteren Verlauf der Arbeit mit \textit{Zeitgemäßes} abgekürzt} entstand direkt vor dem Hintergrund des 1. Weltkrieges\footnote{im weiteren Verlauf der Arbeit \textit{1.WW}} und ist extrem skeptisch im Hinblick auf menschliche Entwicklung verstanden als zivilisiatorischen Fortschritt. \textit{Zeitgemäßes} ist dabei auch geprägt von der von Freud entwickelten ambivalenten Haltung zu Kultur und steht stark unter seinem negativen Bild zur Kulturgründung (\cite{lohmann_freud-handbuch:_2013} S.188 f.)
\subsubsection{Zeitgemäßes über Krieg und Tod}
In seinem Text drückt Freud zunächst seine Enttäuschung darüber aus das die "zivilisierten Nationen"\footnote{Freud entwickelt an dieser Stelle schwer rassistische Untertöne} nicht nur einen Krieg gegeneinander führen, sondern in diesem auch grundlegende humanitäre Gebote nicht eingehalten wurden. Vor dem Hintergrund der Schonungslosigkeit des \textit{1.WW} präsentiert Freud zunächst Kulturentwicklung als eine Entwicklung hin zu immer effektiveren Mordwerkzeugen, nur um im zweiten Teil des Textes aufzuzeigen das auch das etablieren von kulturellen Normen das Triebhafte des Menschen nicht vollständig ausblenden kann (ebd.). Für Freud regelt die Kultur zwar das Verlangen zu Töten, dies wird schon in den im vorherigen Kapitel erwähnten, ersten \textit{Tabus} deutlich, allerdings besteht unter dem "Deckmantel" der Kultur weiter der menschliche Trieb zum Destruktiven, der auch das Töten mit einschließt (\cite{mccall_society--gang_2006} S. 161 ff.). \\
Freud argumentiert, dass der Krieg als soziales Ereignis die Normen der Kultur antastet und teilweise überwirft, indem er beispielsweise ein sonst gültiges Tötungsverbot, zumindest dem "Feind" gegenüber, aufhebt. Während eine solchen Ereignisses beobachtet Freud auch eine generelle Aufhebung der kulturell auferlegten Hemmungen. Er spricht in diesem Zusammenhang von \textit{Regression} und meint damit den Übergang in einen archarischeren Zustand des Seelenlebens, in dem, zumindest vermeintlich, kulturelle Restriktionen vorgeherrscht haben. \textit{Regression} im Angesicht der Aufhebung von kulturellen Hemmungen ist bei Freud ein wiederkehrendes Motiv und scheint seine These zu bestätigen das das Triebhafte dem Menschen weiter innewohnt und nur durch kulturelle Normen mediiert wird. Insofern bestätigt sich hier auch noch einmal die oben Bereits erwähnte These Freuds, dass menschliche Entwicklung nicht als Prozess gesehen werden kann, indem die triebhafte Natur vollständig abgelegt wird (\cite{lohmann_freud-handbuch:_2013} S.188 f.)\\
\subsubsection{Freud und der Tod}
  Im zweiten Teil seines Essays führt Freud den Tod als den nächsten Verwandten des Krieges ein. Er konstatiert dabei eine Zwiegespaltenheit des Menschen im Angesicht des Todes. Einerseits sind sich die Menschen darüber im klaren, dass sie sterben müssen:
\begin{quote} "[...] jeder von uns der Natur einen Tod schulde."
 %Freud 1974 S.49
\end{quote}
Andererseits, ist das Unbewusste überzeugt von der eignen Unsterblichkeit.\\
Freud argumentiert, dass die westliche Kultur es weitgehend verbietet über den Tod, insbesondere den Tod anderer zu sprechen und sich die so sozialisierten Menschen auch nicht gerne dem Gedanken an den Tod hingeben, besonders nicht wenn der mögliche Tod eines anderen mit einem persönlichem Gewinn (Erbe, etc.) verbunden ist. Im Falle des Todes anderer, so Freud weiter, wird vor allem die zufällige Natur des Todes betont (Unfall, hohes Alter, Krankheit, usw.). Für ihn wird der Tod in der westlichen Welt damit von einer faktischen Notwendigkeit des Lebens zu einer Zufälligkeit herabgerückt.
%Freud 1974 S.49 ff.
Freud argumentiert weiter, dass gerade diese Verdrängung des Todes aber das Leben als sinnliches Erlebniss verarmen lässt. Er ist der Ansicht, dass die Vermeidung des Todes mit einer Vermeidung von Gefahren einhergeht die allerdings notwendig für die menschliche Entwicklung sind. Freud verdeutlicht seine Argumentation am Beispiel des Theaters. Im Theater können Menschen mit dem sich in tödlicher Gefahr befindlichem Helden mitleiden, ohne sich dafür selbst in Gefahr begeben zu müssen. \\
%Freud 1074 S. 50 ff. 
Verschiedene AutorInnen haben an dieser Stelle darauf hingewiesen, dass die vorliegende Argumentation relativ unvermittelt eingeführt wird und in Widerspruch zu weiten Teilen der restlichen freudschen Theorie steht (\cite{lohmann_freud-handbuch:_2013} S. 188; \cite{liran_razinsky_how_2009} S. 76 ff.).  Beide genannten Autor\_innen stellen zur Diskussion, ob Freuds an dieser Stelle den Tod glorifiziert. Razinsky versucht sich dabei an einer Interpretation unter Zuhilfenahme der Hegel Lesart von Bataille und argumentiert, dass die Begegnung mit dem Tod beispielsweise im Ritual oder im Theater, einen wichtigen Zwischenraum zwischen dem Leben und dem Tod (verstanden als nicht-Leben) darstellt und somit eine gewisse stimulierende Wirkung auf emotionaler Ebene hat (\cite{liran_razinsky_how_2009} S.76 ff.).\\
Etwas anders argumentiert Lohmann bei seiner Interpretation der betreffenden Stelle. Er sieht Freuds Bemerkung, das die Abwesenheit des Tods das Leben langweilig macht eher als Versuch das Spannungsfeldes zwischen dem kulturellen Verbot ümit denen der Tod belegt ist und die gleichzeitige Ambivalenz im Angesicht des Todes in EInklang zu bringen (\cite{lohmann_freud-handbuch:_2013} S. 188 ff.).
%Nochmal nachlesen
Zur Erläuterung der Ambivalenz des Individuums im Angesicht des Todes, lässt Freud seine Kunstfigur des Urmenschen erneut auftreten (\cite{lohmann_freud-handbuch:_2013} S. 189; 
%Freud 1974 S. 52 ff.
). 
\begin{quote}
\glqq Der Urmensch hat sich in sehr merkwürdiger Weise zum Tode eingestellt. Gar nicht einheitlich, vielmehr recht widerspruchsvoll. Er hat einerseits den Tod ernst genommen, ihn als Aufhebung des Lebens anerkannt und sich seiner in diesem Sinne bedient, anderseits aber auch den Tod geleugnet, ihn zu nichts herabgedrückt. Dieser Widerspruch wurde durch den Umstand ermöglicht, daß er zum Tode des anderen, des Fremden, des Feindes eine radikal andere Stellung einnahm als zu seinem eigenen. Der Tod des anderen war ihm recht, galt ihm als Vernichtung des Verhaßten, und der Urmensch kannte kein Bedenken, ihn herbeizuführen. Er war gewiß ein sehr leidenschaftliches Wesen, grausamer und bösartiger als andere Tiere. Er mordete gerne und wie selbstverständlich. Den Instinkt, der andere Tiere davon abhalten soll, Wesen der gleichen Art zu töten und zu verzehren, brauchen wir ihm nicht zuzuschreiben.\grqq
%Zitat finden (126)
\end{quote}
Dabei wiederholt Freud auch das schon aus "\textit{Totem und Tabu}" bekannte Postulat, dass Geschichte immer auch die Geschichte des Mordes ist. In ihr lässt Freud seinen Urmenschen zunächst als Krieger auftreten, der den Tod für sich selbst verneint ihn aber selbstverständlich Fremden gegenüber herbeiführt. Der Krieger erscheint somit als reuloser Mörder, der sich seiner eigenen Sterblichkeit nicht bewusst ist. 
Allerdings führt Freud führt Freud an selbiger stelle das \textit{Ur-Trauma} ein, das in dem Moment entsteht, als Freuds \textit{Ur-Mensch} als Krieger das erste den Tod einer geliebten Person beobachtet. Einerseits wird die geliebte Person im Unterbewusstsein der \textit{Ur-Mensch} als Teil der eignen Person verstanden und wird dabei vom eigenen Unterbewusstsein, das von seiner Unsterblichkeit überzeugt ist, ebenfalls als Unsterblich betrachtet. Andererseits wird die geliebte Person auch als etwas Fremdes wahrgenommen und der Person wird mit Aggression begegnet. Ambivalenz entsteht daher in diesem Moment als Spannungsfeld zwischen dem bedrohlichen Fremden, das vernichtet werden kann und dem Teil des Selbst das unsterblich ist, das sich in der (toten) geliebten Person manifestiert.
%Freud 1974 52 ff.
Das von Freud angesprochene \textit{Ur-Trauma} lässt den archarischen Krieger einen Kompromiss eingehen
\begin{quote}
"So ließ er sich auf Kompromisse ein, gab den Tod auch für sich zu, bestritt ihm aber die Bedeutung der Lebensvernichtung, wofür ihm beim Tode des Feindes jedes Motiv gefehlt hatte. An der Leiche der geliebten Person ersann er die Geister, und sein Schuldbewußtsein ob der Befriedigung, die der Trauer beigemengt war, bewirkte, daß diese erstgeschaffenen Geister böse Dämonen wurden, vor denen man sich ängstigen mußte." 
%(Freud 1974 S. 54)
\end{quote}
Aus dem Schuldbewusstsein entsteht für Freud auch das erste moralische Gebot der Geschichte:
\begin{quote}
"An der Leiche der geliebten Person entstanden nicht nur die Seelenlehre, der Unsterblichkeitsglaube und eine mächtige Wurzel des menschlichen Schuldbewußtseins, sondern auch die ersten ethischen Gebote. Das erste und bedeutsamste Verbot des erwachenden Gewissens lautete: Du sollst nicht töten. Es war als die Reaktion gegen die hinter der Trauer versteckte Haßbefriedigung am geliebten Toten gewonnen worden, und wurde allmählich auf den ungeliebten Fremden und endlich auch auf den Feind ausgedehn." 
%(Freud 1974 S. 55)
\end{quote}
Freud führt an, dass der archaische Krieger nunmehr nicht als reueloser Mörder gesehen werden kann, sondern, indem er einen Kompromiss eingegangen ist, den Tod weiter verdrängen kann aber ein Schuldbewusstsein entwickelt hat, komplexe Sühneritualle unterlaufen muss sollte er im Kampf getötet haben.
\begin{quote} \glqq
Der Wilde – Australier, Buschmann, Feuerländer – ist keineswegs ein reueloser Mörder; wenn er als Sieger vom Kriegspfade heimkehrt, darf er sein Dorf nicht betreten und sein Weib nicht berühren, ehe er seine kriegerischen Mordtaten durch oft langwierige und mühselige Bußen gesühnt hat. Natürlich liegt die Erklärung aus seinem Aberglauben nahe; der Wilde fürchtet noch die Geisterrache der Erschlagenen. Aber die Geister der erschlagenen Feinde sind nichts anderes als der Ausdruck seines bösen Gewissens ob seiner Blutschuld; hinter diesem Aberglauben verbirgt sich ein Stück ethischer Feinfühligkeit, welches uns Kulturmenschen verloren gegangen ist.\footnote{dazu auch ausführlich in: \glqq Totem und Tabu \grqq}
%Freud 156
\grqq
\end{quote}

Insofern ist Freuds die geschichtliche Entwicklung in zu komplexeren Gesellschaften mit komplexeren sozialen Ordnungen für Freud immer auch eine Geschichte des Todes und des Mordes, da im Angesicht des Todes erst die soziale Ordnung enstanden ist. Im Zuge der geschichtlichen Entwicklung haben sich zwar soziale Ordnungen ausdifferenziert haben, das Unterbewusstsein des heutigen Menschen im Hinblick auf den Tod aber dem des \textit{Ur-Menschen} gleicht. Freud attestiert auch dem heutigen Unterbewusstsein, dass es den eignen Tod nicht verstehen kann:
\begin{quote}
"Was wir unser »Unbewußtes« heißen, die tiefsten, aus Triebregungen bestehenden Schichten unserer Seele, kennt überhaupt nichts Negatives, keine Verneinung – Gegensätze fallen in ihm zusammen – und kennt darum auch nicht den eigenen Tod, dem wir nur einen negativen Inhalt geben können. Dem Todesglauben kommt also nichts Triebhaftes in uns entgegen." 
%(Freud 1974 S. 56)
\end{quote}
 An gleicher Stelle, führt Freud allerdings auch aus, dass auch heute der Tod anderer von einer Ambivalenz gekennzeichnet ist:
 \begin{quote}
 "Anderseits anerkennen wir den Tod für Fremde und Feinde und verhängen ihn über sie ebenso bereitwillig und unbedenklich wie der Urmensch. Hier zeigt sich freilich ein Unterschied, den man in der Wirklichkeit für entscheidend erklären wird. Unser Unbewußtes führt die Tötung nicht aus, es denkt und wünscht sie bloß. Aber es wäre unrecht, diese psychische Realität im Vergleiche zur faktischen so ganz zu unterschätzen. Sie ist bedeutsam und folgenschwer genug. Wir beseitigen in unseren unbewußten Regungen täglich und stündlich alle, die uns im Wege stehen, die uns beleidigt und geschädigt haben. Das »Hol' ihn der Teufel«, das sich so häufig in scherzendem Unmut über unsere Lippen drängt, und das eigentlich sagen will: Hol' ihn der Tod, in unserem Unbewußten ist es ernsthafter, kraftvoller Todeswunsch. Ja, unser Unbewußtes mordet selbst für Kleinigkeiten; [...]" 
 %(Freud 1974 S. 57)
 \end{quote}
 Freud hält fest, dass in der (post-)moderne der Tod als Fakt anerkannt wird, unser Unterbewusstsein den eigenen Tod aber nicht für das Individuum zuzulassen kann.Zwar ist der eigene Tod, wörtlich, unvorstellbar, allerdings ist das Unterbewusstsein ist Fremden, auch geliebten Personen gegenüber, durchaus mordlustig. Freud schließt damit zu argumentieren, dass Tod der Platz in der Gesellschaft gegeben werden muss der ihm zusteht anstatt weiter psychologisch über unsere Verhältnisse zu leben.
 % (Freud 1974 S. 59f.) 
\subsubsection{Die Unvorstellbarkeit des Todes}
Freud erwähnt an mehreren Stellen seines Werks die Unvorstellbarkeit des eigenen Todes. Angesichts der in Kapitel 2 angesprochenen Bereinigung des Schlachtfeldes als Charakteristikum des post-modernen Krieges erscheint es sinnvoll auf diese Unvorstellbarkeit etwas näher einzugehen. \\
Die Unvorstellbarkeit des eignen Todes ist von mehreren Autor\_innen aufgegriffen worden, und bilded eine der Grundlagen bei der Interpretation von "Zeitgemäßes" (vgl. u.a.: \cite{liran_razinsky_how_2009}; \cite{mccall_society--gang_2006}; \cite{stonebridge_what_2009} ) und gilt als eine der bedeutenden Stellen im Text. McCall bemerkt das sich das Unbewusste nicht entwickelt hätte, wäre der eigene Tod für den "Urmenschen" verstehbar gewesen:
\begin{quote}
\textit{If one could “look death in the face,” the passage implies, the unconscious would
 never have arisen.} -  \cite{mccall_society--gang_2006}, S. 265
 \end{quote}
 Freud selber weist darauf hin, dass der Tod für das Individuum nicht nur im übertragenen sinne, sondern auch im wörtlichen unvorstellbar ist. 
 \begin{quote}
 \textit{Der eigene Tod ist ja auch unvorstellbar, und so oft wir den Versuch dazu machen, können wir bemerken, daß wir eigentlich als Zuschauer weiter dabei bleiben. So konnte in der psychoanalytischen Schule der Ausspruch gewagt werden: Im Grunde glaube niemand an seinen eigenen Tod oder, was dasselbe ist: Im Unbewußten sei jeder von uns von seiner Unsterblichkeit überzeugt.} 
 %Freud 110
 \end{quote}
 Der eigene Tod entzieht sich der Repräsentation durch das Individuum, da das Individuum als Zuschauer weiter am eignen Tod teilnimmt. Dabei verliert der Tod seine wichtigste Eigenschaft, nämlich die Vernichtung des Lebens. Der Tod kann daher zwar als Fakt anerkannt werden allerdings bleibt er in seiner Funktion als Vernichtung des Individuums und damit auch seiner Sinne und seines Vorstellungsvermögens, unvorstellbar. Bei der Repräsentation des eigenen Todes bleibt das Individuum aber weiter als Zuschauer anwesend. Die Repräsentation des eignen Todes ist somit immer ein Spektakel das den Zuschauer erfordert und genau deshalb den Tod in seiner vollen Bedeutung unvorstellbar belieben lässt und somit daher immer auch Romantisierung des eigenen Todes darstellt (\cite{weber_wartime_1997} S. 96).\\
 Die Repräsentation des Todes in der heutigen Gesellschaft findet daher meist als Tod des "Anderen" innerhalb der Fiktion (bspw. dem Theater) statt (\cite{liran_razinsky_how_2009} S. 1 ff.). Stonebridge weist darauf hin, das gerade die Unvorstellbarkeit und Unrepräsentierbarkeit des eigenen Todes die Repräsentation als den Tod des anderen wichtig werden lässt, da die Repräsentation in der Fiktion eine Art Versöhnung mit dem eigenen Tod darstellt, da innerhalb der Fiktion dem einen Sinn gegeben werden kann, was sonst als vollständig unverständlich erscheint (\cite{stonebridge_what_2009} S. 103). \\
An dieser Stelle soll noch einmal auf das bereits im vorhergehenden Kapitel erwähnte Freud Zitat eigegangen werden:
\begin{quote}
\textit{So ließ er sich auf Kompromisse ein, gab den Tod auch für sich zu, bestritt ihm aber die Bedeutung der Lebensvernichtung, wofür ihm beim Tode des Feindes jedes Motiv gefehlt hatte. }
%(Freud 1974 S. 54)
\end{quote}
Stonebridge argumentiert, das sich der Kompromiss den Freud hier aanspricht nicht nur auf die geleibte Person, sondern auch auf das selbst anwenden lässt. Die Verneinung des Tods in seiner bedeutung als vernichtung der eignen Existenz wird auch aufrecht erhalten, indem der Tod dem Fremden zugeschoben wird (\cite{stonebridge_what_2009} S. 103 f.). Geschieht dies zumeist in der Fiktion (vgl. bspw. \cite{liran_razinsky_how_2009}) Umgehung des Todes im Krieg nicht länger aufrecht erhalten. 
\begin{quote}
\textit{Es ist evident, daß der Krieg diese konventionelle Behandlung des Todes hinwegfegen muß. Der Tod läßt sich jetzt nicht mehr verleugnen; man muß an ihn glauben. Die Menschen sterben wirklich, auch nicht mehr einzeln, sondern viele, oft Zehntausende an einem Tag. Er ist auch kein Zufall mehr. Es scheint freilich noch zufällig, ob diese Kugel den einen trifft oder den andern; aber diesen anderen mag leicht eine zweite Kugel treffen, die Häufung macht dem Eindruck des Zufälligen ein Ende.}
%Freud 120
\end{quote}
Allerdings ist es keinesfalls so, dass der massenhafte Tod im Krieg eine direkte Repräsentation des eignen Todes möglich macht. Vielmehr wird im Krieg der Tod dem Fremden der im Krieg als Feind auftritt zugeschrieben (\cite{stonebridge_what_2009} S. 103 ff.). 
\begin{quote}
\textit{Es ist leicht zu sagen, wie der Krieg in diese Entzweiung eingreift. Er streift uns die späteren Kulturauflagerungen ab und läßt den Urmenschen in uns wieder zum Vorschein kommen. Er zwingt uns wieder, Helden zu sein, die an den eigenen Tod nicht glauben können; er bezeichnet uns die Fremden als Feinde, deren Tod man herbeiführen oder herbeiwünschen soll; er rät uns, uns über den Tod geliebter Personen hinwegzusetzen.}
%Freud 178
\end{quote}
Krieg, in der Freudschen Theorie, führt also nicht dazu dass sich die Ambivalenz des Individuums hinsichtlich des eigenen Todes verändert, sondern Krieg macht lediglich nur besonders deutlich, dass das Individuum seinen eignen Tod bestreitet, indem es den Tod im fremden, feindselig Anderen sieht. Die politischen Gegebenheiten im Krieg sorgen hier dafür, dass \textit{der Andere} als Feind institutionalisiert wird der getötet werden darf. Die Bestreitung der eignen Sterblichkeit geht im Krieg mit einer Aggression dem Feind gegenüber einher, die den Tod selber als etwas repräsentiert das dem Feind zugefügt werden kann. Der Tod wird zu einem Zustand der dem \textit{Anderem} aufgezwungen werden kann, durch taktisch und strategisch geschickte Manöver. Der Krieg wird somit ebenfalls zu einem Spektakel in dem der Tod immer als der Tod der \textit{Anderen} repräsentiert wird und das Individuum lediglich als Zuschauer anwesend ist, aber nicht mit der vollen Konsequenz der eignen Sterblichkeit Konfrontiert wird, da diese weiter unvorstellbar bleibt. (\cite{weber_wartime_1997} S. 103 ff.). Weber bemerkt hier, dass anders als zu Freuds Zeiten, heute die Repräsentation des Krieges und auch des Todes nicht länger an die physische Anwesenheit des Betrachter geknüpft ist, sondern zunehmend in Medien wie dem Fernsehen oder dem Internet erfolgt. Die Repräsentation des fremden Todes ist daher meist auch eine multiple Repräsentation einer Reihe fremder Tode die an verschiedenen Orten stattfinden können (ebd.).




\newpage
\bibliography{Postmoderne_HA_Bib}
\end{document}
