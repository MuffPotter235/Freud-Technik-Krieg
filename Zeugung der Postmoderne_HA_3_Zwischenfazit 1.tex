\documentclass[11pt,a4paper,oneside,numbers=noenddot,bibliography=totocnumbered,DIV=13]{scrartcl}
%\usepackage[lmargin=2cm,rmargin=4cm,tmargin=3cm,bmargin=3cm]{geometry}
%Sprache
\usepackage[utf8]{inputenc}
\usepackage[T1]{fontenc}
\usepackage[ngerman]{babel}
%Typografie
\usepackage[bitstream-charter]{mathdesign}
%\usepackage{txfonts}
\usepackage[activate={true,nocompatibility},final,tracking=true,kerning=true,spacing=true,factor=1100,stretch=10,shrink=10]{microtype}
\usepackage{url}
%überschriften in Helvetica
\usepackage{sectsty}
\allsectionsfont{\fontfamily{phv}\selectfont\normalsize} 
%Schusterjungen
\clubpenalty = 10000 
\widowpenalty = 10000 
\displaywidowpenalty = 10000
% Zeilenabstand
\usepackage{setspace}
\onehalfspacing
%Zitate
\newcommand{\block}[2]{{\footnotesize\singlespacing\blockquote{\enquote{#1}{#2}}}\vspace{+2mm}}% <-- Hier kannst du den Abstand veraendern
\usepackage{tocstyle} 
%\usetocstyle{KOMAlike}%Find ich eigentlich gut 
\addtocontents{toc}{\protect\fontfamily{phv}} 


\setkomafont{title}{\normalfont\bfseries}


\begin{document}
\section{Zwischenfazit Post-Moderner Krieg}
Im vorangegangenem Kapitel wurde versucht zu zeigen, dass sich die Kriegsführung spätestens seit dem Ende des Kalten Krieges verändert hat. Dabei wird von breiten Teilen der Literatur angenommen, dass sich ein gänzlich  neuer Modus der Kriegsführung, vor allem in den westlichen und Industrienationen, etabliert hat. Dieser Modus  ist zwar keinesfalls alleinstehen, doch, jedenfalls momentan, einer der am weitest Verbreiteten, wenn es zu Anwendung von Militärischer Gewalt durch \footnote{\cite{duyvesteyn_rethinking_2005}} westliche Staaten kommt. Die Literatur hat für diesen Modus unterschiedliche Namen geprägt, beispielweise "\textit{spectacle war}" oder "\textit{globalization war}"\footnote{\cite{bauman_wars_2001}} und schließen sich an eine Dabatte über "Neue Kriege" wie sie beispielweise von Kaldor oder Münkler\footnote{\cite{munkler_neuen_2004}} ausgelöst wurde an. Beide Autor\_innen argumentieren dabei ähnlich, nämlich das es in einem neuen Modus der Kriegsführung nicht primär um die Kontrolle von Territorium, sonder um das durchsetzen von politischen Forderungen möglichst ohne eigne Verluste geht. Die hier verwendete Darstellung dieses Modus von James Der Derian\footnote{\cite{DerDerian2001}}  ist dabei eine der ausführlichsten, da er auf beide Punkte eingeht die diesen neuen Modus kennzeichne und im vorangegangenen Kapitel deutlich gemacht werden sollten.\\
Zum einen werden diese Neuen Kriege durch ethische Imperative begründet und durch die \textit{jus bellum} Doktrin legitimiert, zum anderen ist der Anspruch in diesen Kriegen keine eignen Verluste hinzunehmen, welcher in sich selbst zu einem weiteren ethischen Imperativ wird. Der Derian legt auch sehr gut da wie das was er einen "\textit{technological fix}" nennt angewendet wird um diese beiden, eigentlich gegensätzlichen Imperative zu verbinden. Dabei wird Technologie und neue strategisch/taktische Doktrinen eingesetzt um beiden Imperativen nachzukommen. Von Interesse für diese Arbeit ist dabei nicht primär ob der Einsatz neuer Technologien oder die Verbesserung bereits bestehender tatsächlich in der Lage ist die beiden Imperative zu verbinden, sondern das was von Der Derian mit dem Begriff \textit{technofetishism} beschrieben wird. Dies ist primär der Glaube und das Vertrauen in die Wirkmächtigkeit der Technik und die Möglichkeit in Konflikt stehende Impertaive zu verbinden bzw. die Verbindung durch die Weiterentwicklung der Technik möglich zu machen\footnote{\cite{DerDerian2001}S. 199 ff.}.      
\newpage
\end{document}