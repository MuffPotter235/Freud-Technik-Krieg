\documentclass[11pt,a4paper,oneside,numbers=noenddot,bibliography=totocnumbered,DIV=13]{scrartcl}
%\usepackage[lmargin=2cm,rmargin=4cm,tmargin=3cm,bmargin=3cm]{geometry}
%Sprache
\usepackage[utf8]{inputenc}
\usepackage[T1]{fontenc}
\usepackage[ngerman]{babel}
%Typografie
\usepackage[bitstream-charter]{mathdesign}
%\usepackage{txfonts}
\usepackage[activate={true,nocompatibility},final,tracking=true,kerning=true,spacing=true,factor=1100,stretch=10,shrink=10]{microtype}
\usepackage{url}
%überschriften in Helvetica
\usepackage{sectsty}
\allsectionsfont{\fontfamily{phv}\selectfont\normalsize} 
%Schusterjungen
\clubpenalty = 10000 
\widowpenalty = 10000 
\displaywidowpenalty = 10000
% Zeilenabstand
\usepackage{setspace}
\onehalfspacing
%Zitate
\newcommand{\block}[2]{{\footnotesize\singlespacing\blockquote{\enquote{#1}{#2}}}\vspace{+2mm}}% <-- Hier kannst du den Abstand veraendern
\usepackage{tocstyle} 
%\usetocstyle{KOMAlike}%Find ich eigentlich gut 
\addtocontents{toc}{\protect\fontfamily{phv}} 


\setkomafont{title}{\normalfont\bfseries}


\begin{document}
\section{Sigmund Freud}
Sigmund Freud gilt als Begründer der Psychoanalyse. 
Geboren 1856 in Wien beruht seine Psychoanalyse im wesentlichen auf drei Säulen: einer spezifischen Methodik, einer allgemeinen Metapsychologie sowie einer ebenfalls spezifischen Krankheitslehre. 
Als einer der Gründungstexte dessen was später die Psychoanalyse werden sollte gilt die von Freud in Zusammenarbeit mit Josef Breuer veröffentlichten Texte über die Behandlung von Bertha Pappenheim, besser Bekannt unter dem von Freud und Breuer verwendeten Pseudonym Anna O., durch das was sie \textit{Redekur} nannten.
Freuds große Leistung in der Ausarbeitung der Psychoanalyse, besteht dabei in der wissenschaftlichen Ergründung des Unbewussten (\cite{berkel_sigmund_2008}S. 13 f.).\\
Freud sah sich bei der Entwicklung seiner Theorie und Praxis als strenger Naturwissenschaftler und war skeptisch hinsichtlich allem was er als Vermischung von Philosophie und Naturwissenschaft sah, dennoch machte Freud selber und nach Freud auch andere Autoren die Erkenntnisse der Psychoanalyse nutzbar für die Sozialwissenschaften (\cite{lohmann_freud-handbuch:_2013}S. 10 ff.).
Für diese Arbeit sollen vor allem auf Freuds Theorie zur Kulturentwicklung Bezug genommen werden, insbesondere auf "Zeitgemäßes über Krieg und Tod", da Freud in ihnen die Psychoanalyse auf kollektive Prozesse verallgemeinert hat.        
\subsection{Freuds Kulturtheorie}
Freud entwickelt seine Kulturtheorie ausgehend von den Erkenntnissen der Psychoanalyse. Nach diesen ist das Leben der Menschen immer auf die Befriedigung ihrer triebhaften Bedürfnisse ausgerichtet. 
Allerdings erfolgt diese Triebbefriedigung nicht unmittelbar sondern wird mediert und reglementiert durch die Kultur. Freud geht dabei der Frage nach, wie die notwendigen vergesellschaftungsformen der Kultur, die Zusammenleben als sozialer Verband überhaupt erst möglich machen, in Konflikt stehen  mit dem was er als die (triebhafte) Natur des Menschen begreift (\cite{lohmann_freud-handbuch:_2013} S. 239). \\
Freud kommt bei der Verallgemeinerung seiner klinischen Ergebnisse fast zwangsläufig zu einer Kulturtheorie. 
Er verwendet dabei die Begriffe \textit{Kultur} und \textit{Zivilisation} weitgehend Synonym was zu einer breiten Kulturdefinition führt. 
Für Freud selbst legt dabei folgende Hulturdefinition vor: "all das worin sich das menschliche Leben über seine animalischen Bedingungen erhob"
% Kulturdefnition Freud mit Zitat belegen
Kultur ist daher die Instanz, die menschliches Zusammenleben über die unmittelbare Triebbefriedigung des Individuums hinaus gehen lässt. Freud versucht dabei in seinen Theorien die unbewusste Dimension dieser kollektiven und mit der unmittelbaren Triebentfaltung in Konflikt stehenden, kollektiven Prozesse sichtbar zu machen (\cite{berkel_sigmund_2008}S. 81 ff.)\\
\subsection{Totem und Tabu - Freuds Kulturkritik} 
Freuds kulturantropologische Schrift "Totem und Tabu" ist zwar nicht die erste Äußerung Freuds zur Kultur aber die für diese Arbeit bedeutsamste, da Freud in "Totem und Tabu" den Beginn der Kulturentwicklung nachzuvollziehen versucht ({\cite{berkel_sigmund_2008} S. 83 ff.). \\
Freud untersucht in seiner Schrift die Überscheidungen zwischen "Neurotikern" und nach dem Prinzip des Totemismus organisierte Gesellschaften von dem was er "Primitive" nennt. In beidem findet Freud \textit{Tabus} - starke und ambivalente Moralisch Ge- beziehungsweise Verbote, vor. Im Begriff \textit{Tabu} findet sich, laut Freud, das etwas heilig und Verboten zugleich ist. Er interpretiert das \textit{Tabu} daher als Ausdruck gegensätzlicher Regungen - einerseits des Begehrens und andererseits das Wissen um das verboten sein und die damit zusammenhängenden Schuldgefühle aus dem Begehren dessen was Verboten ist. Bei dem Vergleich mit "primitiven" Gesellschaften, beschäftigt sich Freud mit den zwei, für ihn, grundlegendsten \textit{Tabus} des Totemismus: Dem Inzestverbot und dem damit einhergehenden starken Exogamiegebot sowie dem Verbot das Totemtier ausshalb von rituellen Handlungen zu töten oder zu essen (\cite{berkel_sigmund_2008} S. 83 ff.).\\
Beide \textit{Tabus} sind für Freud grundlegend und beide sind für Freud auch ambivalent. Er argumentiert an mehren Stellen seiner Werke, das dass was nicht Begehrt wird auch nicht Verboten sein müsse. Nach seiner Sicht entspricht daher ein starkes moralisches Verbot auch einem ebenso starken, möglicherweise unbewussten, Verlangen (\cite{lohmann_freud-handbuch:_2013} S. 241 ff.).\\
Heutzutage erscheint es sinnig  die von Freud im Text aufgeworfenen Zusammenhänge als Denkbild und nicht als faktische Realität verstanden werden, da die meisten seiner Grundannahmen wie der Totemismus als vorherrschendes Organisationsprinzip "archaischer" Gesellschaften oder Freuds Vermutung das  kulturelle Ge- und Verbote genetisch von Genration zu Generation weitergegeben werden, als widerlegt gelten und im anbetrachtet zeitgenössischer Forschung nur schwer haltbar sind (\cite{lohmann_freud-handbuch:_2013} S. 169).\\
In seiner Argumentation setzt Freud vergleichbar mit dem Ödipuskomplex des Individuums den Vater analog zum Totemtier. Er geht dabei von einer \textit{Darwinistischen Urhorde} als die erste Gesellschaftsform aus. In dieser \textit{Urhorde} gibt es ein dominantes Männchen, den Vater, der alle Frauen der Horde für sich beansprucht und alle anderen männlichen Nachkommen oder mögliche Nebenbuhler vertreibt. Dem "Vater" kommt damit eine unangefochtene Vormachtstellung in der Gruppe zu. In Freuds Denkbild, taten sich die vertrieben Söhne eines Tages zusammen und erschlugen den übermächtigen Vater. Die Schuldgefühle bei der Tötung der geliebten und trotzdem verhassten Person es Vater, sowie die Vermeidung weiter, ähnlicher Konflikte, stellen für Freud die erste moralische Ordnung da (\cite{lohmann_freud-handbuch:_2013} S. 168 f.).\\
In der rituellen Tötung des Totemtieres und den damit verbundenen Sühneritualen wird, so Freud, die Tötung des Vaters immer wieder aufs neue begangen und gesühnt ohne das dies in der faktischen Tat mündet. Das Inzesttabu ergibt sich in diesem Bild aus der Vermeidung eine neue Vaterfigur zu erschaffen und so die im (rituellen) Vatermord verbundene Gesellschaft erneut zu spalten.  Insofern ist der Mord am Vater für Freud nicht nur auf Individueller sondern auch auf kollektiver Ebene bedeutsam und stellt den Beginn der Sittlichkeit dar (ebd.).\\
Freud argumentiert, dass sich in modernen Gesellschaften die beiden grundlegenden \textit{Tabus} weiter ausdifferenziert und zuletzt in eine komplexe soziale Ordnung übergegangen sind, letztendlich aber nach dem gleichen Schema wie die grundlegenden \textit{Tabus} funktionieren. Die Schuldgefühle im Angesicht des Vatermordes erklären für Freud dann auch die Ambivalenz von \textit{Tabus}, seien diese in Form von verrechtlicher oder moralischer sozialer Ordnung, in der tabuisierte Dinge zwar qua Trieb begehrt werden, sich aber auch Schuldgefühle ob des Begehrens dieser Dinge entwickeln. Die soziale Ordnung die Triebentfaltung des Individuums zwar unterdrückt, ermöglicht so erst soziales Zusammenleben indem sie verhindert das die Individuen ihren Trieben ungehemmt nachgehen (\cite{berkel_sigmund_2008} S. 90 f.).     
\subsection{Freud, Krieg und Tod}
Im vorherigen Kapitel wurde der Vatermord an den Anfang der Kulturentwicklung gesetzt. In Freuds Äußerungen zum Krieg wird vor allem das Thema \textit{Ambivalenz} erneut aufgegriffen, die Schulgefühle im Angesicht des Vatermordes aber zumeist durch die Unnvorstellbarkeit des eigenen Todes ersetzt.\\
Explizite Äußerungen Freuds zum Thema Krieg als kulturelles Ereignis finden sich vor allem in "Zeitgemäßes über Krieg und Tod" und im Briefwechsel mit Albert Einstein "Warum Krieg". In dieser Arbeit soll sich vor allem auf "Zeitgemäßes über Krieg und Tod" bezogen werden, da diese Quelle die Ergiebigste zu sein scheint.\\
"Zeitgemäßes über Krieg und Tod"\footnote{im weiteren Verlauf der Arbeit mit \textit{Zeitgemäßes} abgekürzt} entstand direkt vor dem Hintergrund des 1. Weltkrieges\footnote{im weiteren Verlauf der Arbeit \textit{1.WW}} und ist extrem skeptisch im Hinblick auf menschliche Entwicklung verstanden als zivilisiatorischen Fortschritt. \textit{Zeitgemäßes} ist dabei auch geprägt von der von Freud entwickelten ambivalenten Haltung zu Kultur und steht stark unter seinem negativen Bild zur Kulturgründung (\cite{lohmann_freud-handbuch:_2013} S.188 f.)
\subsubsection{"Zeitgemäßes über Krieg und Tod"}
In seinem Text drückt Freud zunächst seine Enttäuschung darüber aus das die "zivilisierten Nationen"\footnote{Freud entwickelt an dieser Stelle schwer rassistische Untertöne} nicht nur einen Krieg gegeneinander führen, sondern in diesem auch grundlegende humanitäre Gebote nicht eingehalten wurden. Vor dem Hintergrund der Schonungslosigkeit des \textit{1.WW} präsentiert Freud zunächst Kulturentwicklung als eine Entwicklung hin zu immer effektiveren Mordwerkzeugen, nur um im zweiten Teil des Textes aufzuzeigen das auch das etablieren von kulturellen Normen das Triebhafte des Menschen nicht vollständig ausblenden kann (ebd.). Für Freud regelt die Kultur zwar das Verlangen zu Töten, dies wird schon in den im vorherigen Kapitel erwähnten, ersten \textit{Tabus} deutlich, allerdings besteht unter dem "Deckmantel" der Kultur weiter der menschliche Trieb zum Destruktiven, der auch das Töten mit einschließt (\cite{mccall_society--gang_2006} S. 161 ff.). \\
Freud argumentiert, dass der Krieg als soziales Ereignis die Normen der Kultur antastet und teilweise überwirft, indem er beispielsweise ein sonst gültiges Tötungsverbot, zumindest dem "Feind" gegenüber, aufhebt. Während eine solchen Ereignisses beobachtet Freud auch eine generelle Aufhebung der kulturell auferlegten Hemmungen. Er spricht in diesem Zusammenhang von \textit{Regression} und meint damit den Übergang in einen archarischeren Zustand des Seelenlebens, in dem, zumindest vermeintlich, kulturelle Restriktionen vorgeherrscht haben. \textit{Regression} im Angesicht der Aufhebung von kulturellen Hemmungen ist bei Freud ein wiederkehrendes Motiv und scheint seine These zu bestätigen das das Triebhafte dem Menschen weiter innewohnt und nur durch kulturelle Normen mediiert wird. Insofern bestätigt sich hier auch noch einmal die oben Bereits erwähnte These Freuds, dass menschliche Entwicklung nicht als Prozess gesehen werden kann, indem die triebhafte Natur vollständig abgelegt wird (\cite{lohmann_freud-handbuch:_2013} S.188 f.)\\
\subsubsection{"Freud und der Tod"}
  Im zweiten Teil seines Essays führt Freud den Tod als den nächsten Verwandten des Krieges ein. Er konstatiert dabei eine Zwiegespaltenheit des Menschen im Angesicht des Todes. Einerseits sind sich die Menschen darüber im klaren, dass sie sterben müssen:
\begin{quote} "[...] jeder von uns der Natur einen Tod schulde."
 %Freud 1974 S.49
\end{quote}
Andererseits, ist das Unbewusste überzeugt von der eignen Unsterblichkeit.\\
Freud argumentiert, dass die westliche Kultur es weitgehend verbietet über den Tod, insbesondere den Tod anderer zu sprechen und sich die so sozialisierten Menschen auch nicht gerne dem Gedanken an den Tod hingeben, besonders nicht wenn der mögliche Tod eines anderen mit einem persönlichem Gewinn (Erbe, etc.) verbunden ist. Im Falle des Todes anderer, so Freud weiter, wird vor allem die zufällige Natur des Todes betont (Unfall, hohes Alter, Krankheit, usw.). Für ihn wird der Tod in der westlichen Welt damit von einer faktischen Notwendigkeit des Lebens zu einer Zufälligkeit herabgerückt.
%Freud 1974 S.49 ff.
Freud argumentiert weiter, dass gerade diese Verdrängung des Todes aber das Leben als sinnliches Erlebniss verarmen lässt. Er ist der Ansicht, dass die Vermeidung des Todes mit einer Vermeidung von Gefahren einhergeht die allerdings notwendig für die menschliche Entwicklung sind. Freud verdeutlicht seine Argumentation am Beispiel des Theaters. Im Theater können Menschen mit dem sich in tödlicher Gefahr befindlichem Helden mitleiden, ohne sich dafür selbst in Gefahr begeben zu müssen. \\
%Freud 1074 S. 50 ff. 
Verschiedene AutorInnen haben an dieser Stelle darauf hingewiesen, dass die vorliegende Argumentation relativ unvermittelt eingeführt wird und in Widerspruch zu weiten Teilen der restlichen freudschen Theorie steht (\cite{lohmann_freud-handbuch:_2013} S. 188; \cite{liran_razinsky_how_2009} S. 76 ff.).  Beide genannten Autor\_innen stellen zur Diskussion, ob Freuds an dieser Stelle den Tod glorifiziert. Razinsky versucht sich dabei an einer Interpretation unter Zuhilfenahme der Hegel Lesart von Bataille und argumentiert, dass die Begegnung mit dem Tod beispielsweise im Ritual oder im Theater, einen wichtigen Zwischenraum zwischen dem Leben und dem Tod (verstanden als nicht-Leben) darstellt und somit eine gewisse stimulierende Wirkung auf emotionaler Ebene hat (\cite{liran_razinsky_how_2009} S.76 ff.).\\
Etwas anders argumentiert Lohmann bei seiner Interpretation der betreffenden Stelle. Er sieht Freuds Bemerkung, das die Abwesenheit des Tods das Leben langweilig macht eher als weitere Schilderung des Spannungsfeldes zwischen dem kulturellen Verbot über den Tod zu sprechen und die gleichzeitige Ambivalenz hinsichtlich des Todes (\cite{lohmann_freud-handbuch:_2013} S. 188 ff.)
%Nochmal nachlesen
\end{document}