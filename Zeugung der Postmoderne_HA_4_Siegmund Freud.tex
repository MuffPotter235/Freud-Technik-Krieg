\documentclass[11pt,a4paper,oneside,numbers=noenddot,bibliography=totocnumbered,DIV=13]{scrartcl}
%\usepackage[lmargin=2cm,rmargin=4cm,tmargin=3cm,bmargin=3cm]{geometry}
%Sprache
\usepackage[utf8]{inputenc}
\usepackage[T1]{fontenc}
\usepackage[ngerman]{babel}
%Typografie
\usepackage[bitstream-charter]{mathdesign}
%\usepackage{txfonts}
\usepackage[activate={true,nocompatibility},final,tracking=true,kerning=true,spacing=true,factor=1100,stretch=10,shrink=10]{microtype}
\usepackage{url}
%überschriften in Helvetica
\usepackage{sectsty}
\allsectionsfont{\fontfamily{phv}\selectfont\normalsize} 
%Schusterjungen
\clubpenalty = 10000 
\widowpenalty = 10000 
\displaywidowpenalty = 10000
% Zeilenabstand
\usepackage{setspace}
\onehalfspacing
%Zitate
\newcommand{\block}[2]{{\footnotesize\singlespacing\blockquote{\enquote{#1}{#2}}}\vspace{+2mm}}% <-- Hier kannst du den Abstand veraendern
\usepackage{tocstyle} 
%\usetocstyle{KOMAlike}%Find ich eigentlich gut 
\addtocontents{toc}{\protect\fontfamily{phv}} 


\setkomafont{title}{\normalfont\bfseries}


\begin{document}
\section{Sigmund Freud}
Sigmund Freud gilt als Begründer der Psychoanalyse. 
Geboren 1856 in Wien beruht seine Psychoanalyse im wesentlichen auf drei Säulen: einer spezifischen Methodik, einer allgemeinen Metapsychologie sowie einer ebenfalls spezifischen Krankheitslehre. 
Als einer der Gründungstexte dessen was später die Psychoanalyse werden sollte gilt die von Freud in Zusammenarbeit mit Josef Breuer veröffentlichten Texte über die Behandlung von Bertha Pappenheim, besser Bekannt unter dem von Freud und Breuer verwendeten Pseudonym Anna O., durch das was sie \textit{Redekur} nannten.
Freuds große Leistung in der Ausarbeitung der Psychoanalyse, besteht dabei in der wissenschaftlichen Ergründung des Unbewussten (\cite{berkel_sigmund_2008}S. 13 f.).\\
Freud sah sich bei der Entwicklung seiner Theorie und Praxis als strenger Naturwissenschaftler und war skeptisch hinsichtlich allem was er als Vermischung von Philosophie und Naturwissenschaft sah, dennoch machte Freud selber und nach Freud auch andere Autoren die Erkenntnisse der Psychoanalyse nutzbar für die Sozialwissenschaften (\cite{lohmann_freud-handbuch:_2013}S. 10 ff.).
Für diese Arbeit sollen vor allem auf Freuds Theorie zur Kulturentwicklung Bezug genommen werden, insbesondere auf "Zeitgemäßes über Krieg und Tod", da Freud in ihnen die Psychoanalyse auf kollektive Prozesse verallgemeinert hat.        
\subsection{Freuds Kulturtheorie}
Freud entwickelt seine Kulturtheorie ausgehend von den Erkenntnissen der Psychoanalyse. Nach diesen ist das Leben der Menschen immer auf die Befriedigung ihrer triebhaften Bedürfnisse ausgerichtet. 
Allerdings erfolgt diese Triebbefriedigung nicht unmittelbar sondern wird mediert und reglementiert durch die Kultur. Freud geht dabei der Frage nach, wie die notwendigen vergesellschaftungsformen der Kultur, die Zusammenleben als sozialer Verband überhaupt erst möglich machen, in Konflikt stehen  mit dem was er als die (triebhafte) Natur des Menschen begreift (\cite{lohmann_freud-handbuch:_2013} S. 239). \\
Freud kommt bei der Verallgemeinerung seiner klinischen Ergebnisse fast zwangsläufig zu einer Kulturtheorie. 
Er verwendet dabei die Begriffe \textit{Kultur} und \textit{Zivilisation} weitgehend Synonym was zu einer breiten Kulturdefinition führt. 
Für Freud selbst legt dabei folgende Hulturdefinition vor: "all das worin sich das menschliche Leben über seine animalischen Bedingungen erhob"
% Kulturdefnition Freud mit Zitat belegen
Kultur ist daher die Instanz, die menschliches Zusammenleben über die unmittelbare Triebbefriedigung des Individuums hinaus gehen lässt. Freud versucht dabei in seinen Theorien die unbewusste Dimension dieser kollektiven und mit der unmittelbaren Triebentfaltung in Konflikt stehenden, kollektiven Prozesse sichtbar zu machen (\cite{berkel_sigmund_2008}S. 81 ff.)\\
\subsection{Der Mord als Beginn der sozialen Ordnung} 
Freuds kulturantropologische Schrift "Totem und Tabu" ist zwar nicht die erste Äußerung Freuds zur Kultur aber die für diese Arbeit bedeutsamste, da Freud in "Totem und Tabu" den Beginn der Kulturentwicklung nachzuvollziehen versucht ({\cite{berkel_sigmund_2008} S. 83 ff.). \\
Freud untersucht in seiner Schrift die Überscheidungen zwischen \glqq Neurotikern \grqq und nach dem Prinzip des Totemismus organisierte Gesellschaften von dem was er "Primitive" nennt. In beidem findet Freud \textit{Tabus} - starke und ambivalente Moralisch Ge- beziehungsweise Verbote, vor. Im Begriff \textit{Tabu} findet sich, laut Freud, das etwas heilig und Verboten zugleich ist. Er interpretiert das \textit{Tabu} daher als Ausdruck gegensätzlicher Regungen - einerseits des Begehrens und andererseits das Wissen um das verboten sein und die damit zusammenhängenden Schuldgefühle aus dem Begehren dessen was Verboten ist. Bei dem Vergleich mit "primitiven" Gesellschaften, beschäftigt sich Freud mit den zwei, für ihn, grundlegendsten \textit{Tabus} des Totemismus: Dem Inzestverbot und dem damit einhergehenden starken Exogamiegebot sowie dem Verbot das Totemtier ausshalb von rituellen Handlungen zu töten oder zu essen (\cite{berkel_sigmund_2008} S. 83 ff.).\\
Beide \textit{Tabus} sind für Freud grundlegend und beide sind für Freud auch ambivalent. Er argumentiert an mehren Stellen seiner Werke, das dass was nicht Begehrt wird auch nicht Verboten sein müsse. Nach seiner Sicht entspricht daher ein starkes moralisches Verbot auch einem ebenso starken, möglicherweise unbewussten, Verlangen (\cite{lohmann_freud-handbuch:_2013} S. 241 ff.).\\
Heutzutage erscheint es sinnig  die von Freud im Text aufgeworfenen Zusammenhänge als Denkbild und nicht als faktische Realität verstanden werden, da die meisten seiner Grundannahmen wie der Totemismus als vorherrschendes Organisationsprinzip "archaischer" Gesellschaften oder Freuds Vermutung das  kulturelle Ge- und Verbote genetisch von Genration zu Generation weitergegeben werden, als widerlegt gelten und im anbetrachtet zeitgenössischer Forschung nur schwer haltbar sind (\cite{lohmann_freud-handbuch:_2013} S. 169).\\
In seiner Argumentation setzt Freud vergleichbar mit dem Ödipuskomplex des Individuums den Vater analog zum Totemtier. Er geht dabei von einer \textit{Darwinistischen Urhorde} als die erste Gesellschaftsform aus. In dieser \textit{Urhorde} gibt es ein dominantes Männchen, den Vater, der alle Frauen der Horde für sich beansprucht und alle anderen männlichen Nachkommen oder mögliche Nebenbuhler vertreibt. Dem "Vater" kommt damit eine unangefochtene Vormachtstellung in der Gruppe zu. In Freuds Denkbild, taten sich die vertrieben Söhne eines Tages zusammen und erschlugen den übermächtigen Vater. Die Schuldgefühle bei der Tötung der geliebten und trotzdem verhassten Person es Vater, sowie die Vermeidung weiter, ähnlicher Konflikte, stellen für Freud die erste moralische Ordnung da (\cite{lohmann_freud-handbuch:_2013} S. 168 f.).\\
In der rituellen Tötung des Totemtieres und den damit verbundenen Sühneritualen wird, so Freud, die Tötung des Vaters immer wieder aufs neue begangen und gesühnt ohne das dies in der faktischen Tat mündet. Das Inzesttabu ergibt sich in diesem Bild aus der Vermeidung eine neue Vaterfigur zu erschaffen und so die im (rituellen) Vatermord verbundene Gesellschaft erneut zu spalten.  Insofern ist der Mord am Vater für Freud nicht nur auf Individueller sondern auch auf kollektiver Ebene bedeutsam und stellt den Beginn der Sittlichkeit dar (ebd.).\\
Freud argumentiert, dass sich in modernen Gesellschaften die beiden grundlegenden \textit{Tabus} weiter ausdifferenziert und zuletzt in eine komplexe soziale Ordnung übergegangen sind, letztendlich aber nach dem gleichen Schema wie die grundlegenden \textit{Tabus} funktionieren. Die Schuldgefühle im Angesicht des Vatermordes erklären für Freud dann auch die Ambivalenz von \textit{Tabus}, seien diese in Form von verrechtlicher oder moralischer sozialer Ordnung, in der tabuisierte Dinge zwar qua Trieb begehrt werden, sich aber auch Schuldgefühle ob des Begehrens dieser Dinge entwickeln. Die soziale Ordnung die Triebentfaltung des Individuums zwar unterdrückt, ermöglicht so erst soziales Zusammenleben indem sie verhindert das die Individuen ihren Trieben ungehemmt nachgehen (\cite{berkel_sigmund_2008} S. 90 f.).     
\subsection{Freud, Krieg und Tod}
Im vorherigen Kapitel wurde der Vatermord an den Anfang der Kulturentwicklung gesetzt. In Freuds Äußerungen zum Krieg wird vor allem das Thema \textit{Ambivalenz} erneut aufgegriffen, die Schulgefühle im Angesicht des Vatermordes aber zumeist durch die Unnvorstellbarkeit des eigenen Todes ersetzt.\\
Explizite Äußerungen Freuds zum Thema Krieg als kulturelles Ereignis finden sich vor allem in "Zeitgemäßes über Krieg und Tod" und im Briefwechsel mit Albert Einstein "Warum Krieg". In dieser Arbeit soll sich vor allem auf "Zeitgemäßes über Krieg und Tod" bezogen werden, da diese Quelle die Ergiebigste zu sein scheint.\\
"Zeitgemäßes über Krieg und Tod"\footnote{im weiteren Verlauf der Arbeit mit \textit{Zeitgemäßes} abgekürzt} entstand direkt vor dem Hintergrund des 1. Weltkrieges\footnote{im weiteren Verlauf der Arbeit \textit{1.WW}} und ist extrem skeptisch im Hinblick auf menschliche Entwicklung verstanden als zivilisiatorischen Fortschritt. \textit{Zeitgemäßes} ist dabei auch geprägt von der von Freud entwickelten ambivalenten Haltung zu Kultur und steht stark unter seinem negativen Bild zur Kulturgründung (\cite{lohmann_freud-handbuch:_2013} S.188 f.)
\subsubsection{"Zeitgemäßes über Krieg und Tod"}
In seinem Text drückt Freud zunächst seine Enttäuschung darüber aus das die "zivilisierten Nationen"\footnote{Freud entwickelt an dieser Stelle schwer rassistische Untertöne} nicht nur einen Krieg gegeneinander führen, sondern in diesem auch grundlegende humanitäre Gebote nicht eingehalten wurden. Vor dem Hintergrund der Schonungslosigkeit des \textit{1.WW} präsentiert Freud zunächst Kulturentwicklung als eine Entwicklung hin zu immer effektiveren Mordwerkzeugen, nur um im zweiten Teil des Textes aufzuzeigen das auch das etablieren von kulturellen Normen das Triebhafte des Menschen nicht vollständig ausblenden kann (ebd.). Für Freud regelt die Kultur zwar das Verlangen zu Töten, dies wird schon in den im vorherigen Kapitel erwähnten, ersten \textit{Tabus} deutlich, allerdings besteht unter dem "Deckmantel" der Kultur weiter der menschliche Trieb zum Destruktiven, der auch das Töten mit einschließt (\cite{mccall_society--gang_2006} S. 161 ff.). \\
Freud argumentiert, dass der Krieg als soziales Ereignis die Normen der Kultur antastet und teilweise überwirft, indem er beispielsweise ein sonst gültiges Tötungsverbot, zumindest dem "Feind" gegenüber, aufhebt. Während eine solchen Ereignisses beobachtet Freud auch eine generelle Aufhebung der kulturell auferlegten Hemmungen. Er spricht in diesem Zusammenhang von \textit{Regression} und meint damit den Übergang in einen archarischeren Zustand des Seelenlebens, in dem, zumindest vermeintlich, kulturelle Restriktionen vorgeherrscht haben. \textit{Regression} im Angesicht der Aufhebung von kulturellen Hemmungen ist bei Freud ein wiederkehrendes Motiv und scheint seine These zu bestätigen das das Triebhafte dem Menschen weiter innewohnt und nur durch kulturelle Normen mediiert wird. Insofern bestätigt sich hier auch noch einmal die oben Bereits erwähnte These Freuds, dass menschliche Entwicklung nicht als Prozess gesehen werden kann, indem die triebhafte Natur vollständig abgelegt wird (\cite{lohmann_freud-handbuch:_2013} S.188 f.)\\
\subsubsection{"Freud und der Tod"}
  Im zweiten Teil seines Essays führt Freud den Tod als den nächsten Verwandten des Krieges ein. Er konstatiert dabei eine Zwiegespaltenheit des Menschen im Angesicht des Todes. Einerseits sind sich die Menschen darüber im klaren, dass sie sterben müssen:
\begin{quote} "[...] jeder von uns der Natur einen Tod schulde."
 %Freud 1974 S.49
\end{quote}
Andererseits, ist das Unbewusste überzeugt von der eignen Unsterblichkeit.\\
Freud argumentiert, dass die westliche Kultur es weitgehend verbietet über den Tod, insbesondere den Tod anderer zu sprechen und sich die so sozialisierten Menschen auch nicht gerne dem Gedanken an den Tod hingeben, besonders nicht wenn der mögliche Tod eines anderen mit einem persönlichem Gewinn (Erbe, etc.) verbunden ist. Im Falle des Todes anderer, so Freud weiter, wird vor allem die zufällige Natur des Todes betont (Unfall, hohes Alter, Krankheit, usw.). Für ihn wird der Tod in der westlichen Welt damit von einer faktischen Notwendigkeit des Lebens zu einer Zufälligkeit herabgerückt.
%Freud 1974 S.49 ff.
Freud argumentiert weiter, dass gerade diese Verdrängung des Todes aber das Leben als sinnliches Erlebniss verarmen lässt. Er ist der Ansicht, dass die Vermeidung des Todes mit einer Vermeidung von Gefahren einhergeht die allerdings notwendig für die menschliche Entwicklung sind. Freud verdeutlicht seine Argumentation am Beispiel des Theaters. Im Theater können Menschen mit dem sich in tödlicher Gefahr befindlichem Helden mitleiden, ohne sich dafür selbst in Gefahr begeben zu müssen. \\
%Freud 1074 S. 50 ff. 
Verschiedene AutorInnen haben an dieser Stelle darauf hingewiesen, dass die vorliegende Argumentation relativ unvermittelt eingeführt wird und in Widerspruch zu weiten Teilen der restlichen freudschen Theorie steht (\cite{lohmann_freud-handbuch:_2013} S. 188; \cite{liran_razinsky_how_2009} S. 76 ff.).  Beide genannten Autor\_innen stellen zur Diskussion, ob Freuds an dieser Stelle den Tod glorifiziert. Razinsky versucht sich dabei an einer Interpretation unter Zuhilfenahme der Hegel Lesart von Bataille und argumentiert, dass die Begegnung mit dem Tod beispielsweise im Ritual oder im Theater, einen wichtigen Zwischenraum zwischen dem Leben und dem Tod (verstanden als nicht-Leben) darstellt und somit eine gewisse stimulierende Wirkung auf emotionaler Ebene hat (\cite{liran_razinsky_how_2009} S.76 ff.).\\
Etwas anders argumentiert Lohmann bei seiner Interpretation der betreffenden Stelle. Er sieht Freuds Bemerkung, das die Abwesenheit des Tods das Leben langweilig macht eher als Versuch das Spannungsfeldes zwischen dem kulturellen Verbot ümit denen der Tod belegt ist und die gleichzeitige Ambivalenz im Angesicht des Todes in EInklang zu bringen (\cite{lohmann_freud-handbuch:_2013} S. 188 ff.).
%Nochmal nachlesen
Zur Erläuterung der Ambivalenz des Individuums im Angesicht des Todes, lässt Freud seine Kunstfigur des Urmenschen erneut auftreten (\cite{lohmann_freud-handbuch:_2013} S. 189; 
%Freud 1974 S. 52 ff.
). 
\begin{quote}
\glqq Der Urmensch hat sich in sehr merkwürdiger Weise zum Tode eingestellt. Gar nicht einheitlich, vielmehr recht widerspruchsvoll. Er hat einerseits den Tod ernst genommen, ihn als Aufhebung des Lebens anerkannt und sich seiner in diesem Sinne bedient, anderseits aber auch den Tod geleugnet, ihn zu nichts herabgedrückt. Dieser Widerspruch wurde durch den Umstand ermöglicht, daß er zum Tode des anderen, des Fremden, des Feindes eine radikal andere Stellung einnahm als zu seinem eigenen. Der Tod des anderen war ihm recht, galt ihm als Vernichtung des Verhaßten, und der Urmensch kannte kein Bedenken, ihn herbeizuführen. Er war gewiß ein sehr leidenschaftliches Wesen, grausamer und bösartiger als andere Tiere. Er mordete gerne und wie selbstverständlich. Den Instinkt, der andere Tiere davon abhalten soll, Wesen der gleichen Art zu töten und zu verzehren, brauchen wir ihm nicht zuzuschreiben.\grqq
%Zitat finden (126)
\end{quote}
Dabei wiederholt Freud auch das schon aus "\textit{Totem und Tabu}" bekannte Postulat, dass Geschichte immer auch die Geschichte des Mordes ist. In ihr lässt Freud seinen Urmenschen zunächst als Krieger auftreten, der den Tod für sich selbst verneint ihn aber selbstverständlich Fremden gegenüber herbeiführt. Der Krieger erscheint somit als reuloser Mörder, der sich seiner eigenen Sterblichkeit nicht bewusst ist. 
Allerdings führt Freud führt Freud an selbiger stelle das \textit{Ur-Trauma} ein, das in dem Moment entsteht, als Freuds \textit{Ur-Mensch} als Krieger das erste den Tod einer geliebten Person beobachtet. Einerseits wird die geliebte Person im Unterbewusstsein der \textit{Ur-Mensch} als Teil der eignen Person verstanden und wird dabei vom eigenen Unterbewusstsein, das von seiner Unsterblichkeit überzeugt ist, ebenfalls als Unsterblich betrachtet. Andererseits wird die geliebte Person auch als etwas Fremdes wahrgenommen und der Person wird mit Aggression begegnet. Ambivalenz entsteht daher in diesem Moment als Spannungsfeld zwischen dem bedrohlichen Fremden, das vernichtet werden kann und dem Teil des Selbst das unsterblich ist, das sich in der (toten) geliebten Person manifestiert.
%Freud 1974 52 ff.
Das von Freud angesprochene \textit{Ur-Trauma} lässt den archarischen Krieger einen Kompromiss eingehen
\begin{quote}
"So ließ er sich auf Kompromisse ein, gab den Tod auch für sich zu, bestritt ihm aber die Bedeutung der Lebensvernichtung, wofür ihm beim Tode des Feindes jedes Motiv gefehlt hatte. An der Leiche der geliebten Person ersann er die Geister, und sein Schuldbewußtsein ob der Befriedigung, die der Trauer beigemengt war, bewirkte, daß diese erstgeschaffenen Geister böse Dämonen wurden, vor denen man sich ängstigen mußte." 
%(Freud 1974 S. 54)
\end{quote}
Aus dem Schuldbewusstsein entsteht für Freud auch das erste moralische Gebot der Geschichte:
\begin{quote}
"An der Leiche der geliebten Person entstanden nicht nur die Seelenlehre, der Unsterblichkeitsglaube und eine mächtige Wurzel des menschlichen Schuldbewußtseins, sondern auch die ersten ethischen Gebote. Das erste und bedeutsamste Verbot des erwachenden Gewissens lautete: Du sollst nicht töten. Es war als die Reaktion gegen die hinter der Trauer versteckte Haßbefriedigung am geliebten Toten gewonnen worden, und wurde allmählich auf den ungeliebten Fremden und endlich auch auf den Feind ausgedehn." 
%(Freud 1974 S. 55)
\end{quote}
Freud führt an, dass der archaische Krieger nunmehr nicht als reueloser Mörder gesehen werden kann, sondern, indem er einen Kompromiss eingegangen ist, den Tod weiter verdrängen kann aber ein Schuldbewusstsein entwickelt hat, komplexe Sühneritualle unterlaufen muss sollte er im Kampf getötet haben.
\begin{quote} \glqq
Der Wilde – Australier, Buschmann, Feuerländer – ist keineswegs ein reueloser Mörder; wenn er als Sieger vom Kriegspfade heimkehrt, darf er sein Dorf nicht betreten und sein Weib nicht berühren, ehe er seine kriegerischen Mordtaten durch oft langwierige und mühselige Bußen gesühnt hat. Natürlich liegt die Erklärung aus seinem Aberglauben nahe; der Wilde fürchtet noch die Geisterrache der Erschlagenen. Aber die Geister der erschlagenen Feinde sind nichts anderes als der Ausdruck seines bösen Gewissens ob seiner Blutschuld; hinter diesem Aberglauben verbirgt sich ein Stück ethischer Feinfühligkeit, welches uns Kulturmenschen verloren gegangen ist.\footnote{dazu auch ausführlich in: \glqq Totem und Tabu \grqq}
%Freud 156
\grqq
\end{quote}

Insofern ist Freuds die geschichtliche Entwicklung in zu komplexeren Gesellschaften mit komplexeren sozialen Ordnungen für Freud immer auch eine Geschichte des Todes und des Mordes, da im Angesicht des Todes erst die soziale Ordnung enstanden ist. Im Zuge der geschichtlichen Entwicklung haben sich zwar soziale Ordnungen ausdifferenziert haben, das Unterbewusstsein des heutigen Menschen im Hinblick auf den Tod aber dem des \textit{Ur-Menschen} gleicht. Freud attestiert auch dem heutigen Unterbewusstsein, dass es den eignen Tod nicht verstehen kann:
\begin{quote}
"Was wir unser »Unbewußtes« heißen, die tiefsten, aus Triebregungen bestehenden Schichten unserer Seele, kennt überhaupt nichts Negatives, keine Verneinung – Gegensätze fallen in ihm zusammen – und kennt darum auch nicht den eigenen Tod, dem wir nur einen negativen Inhalt geben können. Dem Todesglauben kommt also nichts Triebhaftes in uns entgegen." 
%(Freud 1974 S. 56)
\end{quote}
 An gleicher Stelle, führt Freud allerdings auch aus, dass auch heute der Tod anderer von einer Ambivalenz gekennzeichnet ist:
 \begin{quote}
 "Anderseits anerkennen wir den Tod für Fremde und Feinde und verhängen ihn über sie ebenso bereitwillig und unbedenklich wie der Urmensch. Hier zeigt sich freilich ein Unterschied, den man in der Wirklichkeit für entscheidend erklären wird. Unser Unbewußtes führt die Tötung nicht aus, es denkt und wünscht sie bloß. Aber es wäre unrecht, diese psychische Realität im Vergleiche zur faktischen so ganz zu unterschätzen. Sie ist bedeutsam und folgenschwer genug. Wir beseitigen in unseren unbewußten Regungen täglich und stündlich alle, die uns im Wege stehen, die uns beleidigt und geschädigt haben. Das »Hol' ihn der Teufel«, das sich so häufig in scherzendem Unmut über unsere Lippen drängt, und das eigentlich sagen will: Hol' ihn der Tod, in unserem Unbewußten ist es ernsthafter, kraftvoller Todeswunsch. Ja, unser Unbewußtes mordet selbst für Kleinigkeiten; [...]" 
 %(Freud 1974 S. 57)
 \end{quote}
 Freud hält fest, dass in der (post-)moderne der Tod als Fakt anerkannt wird, unser Unterbewusstsein den eigenen Tod aber nicht für das Individuum zuzulassen kann.Zwar ist der eigene Tod, wörtlich, unvorstellbar, allerdings ist das Unterbewusstsein ist Fremden, auch geliebten Personen gegenüber, durchaus mordlustig. Freud schließt damit zu argumentieren, dass Tod der Platz in der Gesellschaft gegeben werden muss der ihm zusteht anstatt weiter psychologisch über unsere Verhältnisse zu leben.
 % (Freud 1974 S. 59f.) 
\subsubsection{Die Unvorstellbarkeit des Todes}
Freud erwähnt an mehreren Stellen seines Werks die Unvorstellbarkeit des eigenen Todes. Angesichts der in Kapitel 2 angesprochenen Bereinigung des Schlachtfeldes als Charakteristikum des post-modernen Krieges erscheint es sinnvoll auf diese Unvorstellbarkeit etwas näher einzugehen. \\
Die Unvorstellbarkeit des eignen Todes ist von mehreren Autor\_innen aufgegriffen worden, und bilded eine der Grundlagen bei der Interpretation von "Zeitgemäßes" (vgl. u.a.: \cite{liran_razinsky_how_2009}; \cite{mccall_society--gang_2006}; \cite{stonebridge_what_2009} ) und gilt als eine der bedeutenden Stellen im Text. McCall bemerkt das sich das Unbewusste nicht entwickelt hätte, wäre der eigene Tod für den "Urmenschen" verstehbar gewesen:
\begin{quote}
\textit{If one could “look death in the face,” the passage implies, the unconscious would
 never have arisen.} -  \cite{mccall_society--gang_2006}, S. 265
 \end{quote}
 Freud selber weist darauf hin, dass der Tod für das Individuum nicht nur im übertragenen sinne, sondern auch im wörtlichen unvorstellbar ist. 
 \begin{quote}
 \textit{Der eigene Tod ist ja auch unvorstellbar, und so oft wir den Versuch dazu machen, können wir bemerken, daß wir eigentlich als Zuschauer weiter dabei bleiben. So konnte in der psychoanalytischen Schule der Ausspruch gewagt werden: Im Grunde glaube niemand an seinen eigenen Tod oder, was dasselbe ist: Im Unbewußten sei jeder von uns von seiner Unsterblichkeit überzeugt.} 
 %Freud 110
 \end{quote}
 Der eigene Tod entzieht sich der Repräsentation durch das Individuum, da das Individuum als Zuschauer weiter am eignen Tod teilnimmt. Dabei verliert der Tod seine wichtigste Eigenschaft, nämlich die Vernichtung des Lebens. Der Tod kann daher zwar als Fakt anerkannt werden allerdings bleibt er in seiner Funktion als Vernichtung des Individuums und damit auch seiner Sinne und seines Vorstellungsvermögens, unvorstellbar. Bei der Repräsentation des eigenen Todes bleibt das Individuum aber weiter als Zuschauer anwesend. Die Repräsentation des eignen Todes ist somit immer ein Spektakel das den Zuschauer erfordert und genau deshalb den Tod in seiner vollen Bedeutung unvorstellbar belieben lässt und somit daher immer auch Romantisierung des eigenen Todes darstellt (\cite{weber_wartime_1997} S. 96).\\
 Die Repräsentation des Todes in der heutigen Gesellschaft findet daher meist als Tod des "Anderen" innerhalb der Fiktion (bspw. dem Theater) statt (\cite{liran_razinsky_how_2009} S. 1 ff.). Stonebridge weist darauf hin, das gerade die Unvorstellbarkeit und Unrepräsentierbarkeit des eigenen Todes die Repräsentation als den Tod des anderen wichtig werden lässt, da die Repräsentation in der Fiktion eine Art Versöhnung mit dem eigenen Tod darstellt, da innerhalb der Fiktion dem einen Sinn gegeben werden kann, was sonst als vollständig unverständlich erscheint (\cite{stonebridge_what_2009} S. 103). \\
An dieser Stelle soll noch einmal auf das bereits im vorhergehenden Kapitel erwähnte Freud Zitat eigegangen werden:
\begin{quote}
\textit{So ließ er sich auf Kompromisse ein, gab den Tod auch für sich zu, bestritt ihm aber die Bedeutung der Lebensvernichtung, wofür ihm beim Tode des Feindes jedes Motiv gefehlt hatte. }
%(Freud 1974 S. 54)
\end{quote}
Stonebridge argumentiert, das sich der Kompromiss den Freud hier aanspricht nicht nur auf die geleibte Person, sondern auch auf das selbst anwenden lässt. Die Verneinung des Tods in seiner bedeutung als vernichtung der eignen Existenz wird auch aufrecht erhalten, indem der Tod dem Fremden zugeschoben wird (\cite{stonebridge_what_2009} S. 103 f.). Geschieht dies zumeist in der Fiktion (vgl. bspw. \cite{liran_razinsky_how_2009}) Umgehung des Todes im Krieg nicht länger aufrecht erhalten. 
\begin{quote}
\textit{Es ist evident, daß der Krieg diese konventionelle Behandlung des Todes hinwegfegen muß. Der Tod läßt sich jetzt nicht mehr verleugnen; man muß an ihn glauben. Die Menschen sterben wirklich, auch nicht mehr einzeln, sondern viele, oft Zehntausende an einem Tag. Er ist auch kein Zufall mehr. Es scheint freilich noch zufällig, ob diese Kugel den einen trifft oder den andern; aber diesen anderen mag leicht eine zweite Kugel treffen, die Häufung macht dem Eindruck des Zufälligen ein Ende.}
%Freud 120
\end{quote}
Allerdings ist es keinesfalls so, dass der massenhafte Tod im Krieg eine direkte Repräsentation des eignen Todes möglich macht. Vielmehr wird im Krieg der Tod dem Fremden der im Krieg als Feind auftritt zugeschrieben (\cite{stonebridge_what_2009} S. 103 ff.). 
\begin{quote}
\textit{Es ist leicht zu sagen, wie der Krieg in diese Entzweiung eingreift. Er streift uns die späteren Kulturauflagerungen ab und läßt den Urmenschen in uns wieder zum Vorschein kommen. Er zwingt uns wieder, Helden zu sein, die an den eigenen Tod nicht glauben können; er bezeichnet uns die Fremden als Feinde, deren Tod man herbeiführen oder herbeiwünschen soll; er rät uns, uns über den Tod geliebter Personen hinwegzusetzen.}
%Freud 178
\end{quote}
Krieg, in der Freudschen Theorie, führt also nicht dazu dass sich die Ambivalenz des Individuums hinsichtlich des eigenen Todes verändert, sondern Krieg macht lediglich nur besonders deutlich, dass das Individuum seinen eignen Tod bestreitet, indem es den Tod im fremden, feindselig Anderen sieht. Die politischen Gegebenheiten im Krieg sorgen hier dafür, dass \textit{der Andere} als Feind institutionalisiert wird der getötet werden darf. Die Bestreitung der eignen Sterblichkeit geht im Krieg mit einer Aggression dem Feind gegenüber einher, die den Tod selber als etwas repräsentiert das dem Feind zugefügt werden kann. Der Tod wird zu einem Zustand der dem \textit{Anderem} aufgezwungen werden kann, durch taktisch und strategisch geschickte Manöver. Der Krieg wird somit ebenfalls zu einem Spektakel in dem der Tod immer als der Tod der \textit{Anderen} repräsentiert wird und das Individuum lediglich als Zuschauer anwesend ist, aber nicht mit der vollen Konsequenz der eignen Sterblichkeit Konfrontiert wird, da diese weiter unvorstellbar bleibt. (\cite{weber_wartime_1997} S. 103 ff.). Weber bemerkt hier, dass anders als zu Freuds Zeiten, heute die Repräsentation des Krieges und auch des Todes nicht länger an die physische Anwesenheit des Betrachter geknüpft ist, sondern zunehmend in Medien wie dem Fernsehen oder dem Internet erfolgt. Die Repräsentation des fremden Todes ist daher meist auch eine multiple Repräsentation einer Reihe fremder Tode die an verschiedenen Orten stattfinden können (ebd.).
\end{document}